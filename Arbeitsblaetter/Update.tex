% !TeX root = ../Skript_DB.tex
\cohead{\Large\textbf{DELETE/UPDATE-Statement}}
\subsection[UPDATE-Statement]{SQL - Das UPDATE-Statement}\label{update}
Mit Hilfe des UPDATE-Statements lassen sich einzelne Werte eines Eintrags in einer Tabelle nachträglich ändern:
\begin{tcolorbox}[title=UPDATE-Statement]
	\lstinline[breaklines=true]!UPDATE name_der_Tabelle SET attribut1=wert1, attribut2=wert2, ... WHERE bedingungen;!
\end{tcolorbox}
\textcolor{red}{ACHTUNG: Vergisst man die WHERE-Klausel oder wählt mit der WHERE-Klausel mehr Einträge aus als beabsichtigt, werden ALLE oder mehr Einträge als beabsichtigt geändert.}
\begin{Exercise}[title={Ändere folgende Einträge aus der Datenbank:}, label=Update]
	\begin{enumerate}
		\item Der Schüler mit der \lstinline!schuelerNR! 19 soll mit \lstinline!vornamen! Hans Gustav Adalbert heißen.
		\item Die Bezeichnung der \lstinline!klasse! BK2 soll auf BK22 geändert werden.
		\item Bei einigen Schülern ist ein Problem beim Eintragen der \lstinline!plz! aufgetreten. Bei allen Schülern mit einer 4-stelligen \lstinline!plz! soll diese auf NULL geändert werden.
	\end{enumerate}
\end{Exercise}
%%%%%%%%%%%%%%%%%%%%%%%%%%%%%%%%%%%%%%%%%
\begin{Answer}[ref=Update]
	\begin{enumerate}
		\item Der Schüler mit der \lstinline!schuelerNR! 19 soll mit \lstinline!vornamen! Hans Gustav Adalbert heißen.

		\lstinline!UPDATE schueler SET vorname = 'Hans Gustav Adalbert' WHERE schuelerNR = 19;!
		\item Die Bezeichnung der \lstinline!klasse! BK2 soll auf BK22 geändert werden.

		\lstinline!UPDATE schueler SET klasse = 'BK22' WHERE klasse IS 'BK2';!
		\item Bei einigen Schülern ist ein Problem beim Eintragen der \lstinline!plz! aufgetreten. Bei allen Schülern mit einer 4-stelligen \lstinline!plz! soll diese auf NULL geändert werden.

		\lstinline!UPDATE schueler SET plz = NULL WHERE plz < 10000;!
	\end{enumerate}
\end{Answer}