\documentclass[a4paper,12pt, headsepline, ngerman]{scrartcl}

\usepackage{scrlayer-scrpage}
\usepackage[nodisplayskipstretch]{setspace} %vspace before/after math mode
%\setstretch{5}
\raggedbottom
\raggedright
\onehalfspacing
%\renewcommand*{\familydefault}{\sfdefault}



\pagestyle{headings} %KOMA-Script mit Kopf-Fuß-Zeilen
\usepackage{hyperref}			%Hyperlinks setzen
\usepackage[answerdelayed]{exercise}			%Nach hyperref einbinden!
\usepackage{babel}				%Silbentrennung mit ngerman

\usepackage{booktabs} 			% For prettier tables

\usepackage{mathtools}  		%Mathe-Paket
\usepackage{amssymb}			%Symbole
\usepackage{bbm}				%\mathbbm{N} für natürliche Zahlen o.ä.
%\usepackage{graphicx}			%Optionen für \includegraphics{imagefile}
%Solte in mathtools beinhaltet sein
\usepackage{color}				%\textcolor{blue}{text...}
\usepackage[dvipsnames]{xcolor}
%Häufig verwendetet Farben
%ForestGreen 		RGB(  0,155, 85)
%YellowOrange		RGB(250,162, 26)

\usepackage[T1]{fontenc}		%Umlaute
\usepackage[utf8]{inputenc}
\usepackage{wrapfig}
\usepackage{caption}


\usepackage{ulem}
%\uline{important} underlined text
%\uuline{urgent} double-underlined text
%\uwave{boat} wavy underline
%\sout{wrong} line struck through word
%\xout{removed} marked over like //////////
%\dashuline{dashing} dashed underline
%\dotuline{dotty} dotted underline
\usepackage{cancel}					%Durchstreichen von Dingen in Formeln
\usepackage{enumitem}				%Aufzählungen [label=\alph*)]
\setlist[enumerate]{nosep, topsep=0pt}	%Kleinere Abstände bei Aufzählungen
\setlist[itemize]{noitemsep, topsep=0pt}

\usepackage{framed}                 %Rahmen machen \begin{framed} ... \end{framed}

\usepackage{tcolorbox} 				%Für Boxen um Text
\usepackage{array}
%Plots
\usepackage{tikz}
\usepackage{pgf}
\usepackage{pgfmath}

\usepackage{bm}						%\bm{xxx} for bold in math mode

\usepackage{geometry}
\geometry{a4paper, portrait, left=1.5cm, right=2cm, top=1cm, bottom=2cm, headsep=0.2cm, includehead, head=27.30193pt}
\usepackage{listings} %\lstinline[language=C]!while{$a || $b}!
\definecolor{codegreen}{rgb}{0,0.6,0}
\definecolor{codegray}{rgb}{0.5,0.5,0.5}
\definecolor{codepurple}{rgb}{0.58,0,0.82}
\definecolor{backcolour}{rgb}{0.95,0.95,0.92}
\definecolor{basiccolour}{rgb}{0.9,0,0.6}

\lstdefinestyle{mystyle}{
	language=SQL,
	backgroundcolor=\color{backcolour},
	commentstyle=\color{codegreen},
	keywordstyle=\color{magenta},
	numberstyle=\tiny\color{codegray},
	stringstyle=\color{codepurple},
	basicstyle=\color{basiccolour}\ttfamily,
	breakatwhitespace=false,
	breaklines=false,
	captionpos=b,
	keepspaces=false,
	extendedchars=true,
	numbers=left,
	numbersep=5pt,
	showspaces=false,
	showstringspaces=false,
	showtabs=false,
	tabsize=2,
	columns=fullflexible %erzeugt keine komischen Leerzeichen mehr, die man erst beim Kopieren sieht
}
\lstset{literate=%
	{Ö}{{\"O}}1
	{Ä}{{\"A}}1
	{Ü}{{\"U}}1
	{ß}{{\ss}}1
	{ü}{{\"u}}1
	{ä}{{\"a}}1
	{ö}{{\"o}}1
	{~}{{\textasciitilde}}1
}
\lstset{style=mystyle}

\usepackage{marvosym} %\Lightning
\usepackage{multirow}
\renewcommand{\mvchr}[1]{\Large{\mbox{\mvs\symbol{#1}}}} %\Lightning in math mode

\usetikzlibrary{intersections}

\setkomafont{headsepline}{\color{black}}

\usepackage{amsthm}					%Definitionsumgebung für \newtheorem{defi}{Definition}[section] usw.
\theoremstyle{definition}
\newtheorem{defi}{Definition}[subsection]
\newtheorem*{bsp}{Beispiel}
\newtheorem{kon}[defi]{Konstruktion}
\newtheorem{nota}[defi]{Notation}
\newtheorem{cha}[defi]{Charakterisierung}
\newtheorem{norm}[defi]{Normierung}
\newtheorem{bem}[defi]{Bemerkung}
\newtheorem{folg}[defi]{Folgerung}
\newtheorem{beob}[defi]{Beobachtung}
\newtheorem{erin}[defi]{Erinnerung}
\newtheorem{sit}[defi]{Situation}
%Einheiten
\newcommand{\ms}{\frac{m}{s}}
\newcommand{\kmh}{\frac{km}{h}}
%Mathebefehle
\newcommand{\beq}{\begin{align}}
\newcommand{\eeq}{\end{align}}
\newcommand{\beqn}{\begin{align*}}
\newcommand{\eeqn}{\end{align*}}
\newcommand{\td}{\text{d}}
\newcommand{\ul}{\underline}
\newcommand{\tTr}{\text{Tr}}
\newcommand{\bra}[1]{\langle #1|}
\newcommand{\braa}{\langle}
\newcommand{\ket}[1]{|#1\rangle}
\newcommand{\kett}{\rangle}
\newcommand{\braket}[2]{\langle #1|#2\rangle}
\newcommand{\mH}{\mathcal{H}}
\newcommand{\R}{\mathbb{R}}
\newcommand{\N}{\mathbb{N}}
\newcommand{\Z}{\mathbb{Z}}
\newcommand{\Q}{\mathbb{Q}}

%Exercise-Paket Umbenennungen
\renewcommand{\listexercisename}{Liste der Aufgaben}%
\renewcommand{\ExerciseName}{Aufgabe}%
\renewcommand{\AnswerName}{L{\"o}sung zu Aufgabe}%
\renewcommand{\ExerciseListName}{Aufg.}%
\renewcommand{\AnswerListName}{L{\"o}sung}%
\renewcommand{\ExePartName}{Teil}%
\renewcommand{\ArticleOf}{von\ }%
%\renewcommand{\ExerciseHeaderTitle}{\ExerciseTitle}
\renewcommand{\ExerciseHeader}{%
	\textbf{\large\ExerciseHeaderDifficulty\ExerciseName\ %
	\ExerciseHeaderNB\normalsize\ExerciseHeaderTitle\ExerciseHeaderOrigin}\medskip}
\renewcommand{\AnswerHeader}{
	\medskip\textbf{L{\"o}sung zu \ExerciseName\ \ExerciseHeaderNB}\smallskip}


%Farbe für die Lösungen, die die Schüler selbst ausfüllen sollen
\definecolor{loes}{rgb}{1,1,1}
\definecolor{tcback}{rgb}{.95,.95,.95}
\definecolor{tcframe}{rgb}{.89,.15,.21}
\definecolor{loestc}{rgb}{.95,.95,.95}
%Arbeitsblatt-Modus
\definecolor{loes}{rgb}{.36,.58,.93}
\definecolor{loestc}{rgb}{1, .4, 1}
\tcbset{
%	frame code={}
%	center title,
%	left=0pt,
%	right=0pt,
%	top=0pt,
%	bottom=0pt,
	fonttitle=\large\bfseries,
	colback=tcback,
	colframe=tcframe,
%	width=\dimexpr\textwidth\relax,
%	enlarge left by=0mm,
%	boxsep=5pt,
%	arc=0pt,outer arc=0pt,
}
\tcbuselibrary{raster}

%\sqrt[\leftroot{0}\uproot{2}n]{x}
\begin{document}
	\setlength\parindent{0pt} %keine Einrückungen beim Start eines Paragraphen

	%Header
	\lohead{Datenbanken}
	%\cohead{} %im Arbeitsblatt
	\rohead{}
	\lehead{lehead}
	\cehead{cehead}
	\rehead{rehead}
	\cofoot[\pagemark]{\pagemark}
	\title{Datenbanken\\
	Ein Skript für das Berufskolleg}
	\author{Hermann Maier}
	\maketitle
	\thispagestyle{empty}
	\newpage
	\null\vfill
	\copyright 2023 Maier, Hermann, maier(at)privatemail.com\\
	Dieses Werk unterliegt der CC BY-NC-SA 4.0 Lizenz \href{https://creativecommons.org/licenses/by-nc-sa/4.0/legalcode.en}{https://creativecommons.org/licenses/by-nc-sa/4.0/legalcode.en}.\\
	Sie dürfen:\\
	\begin{itemize}
		\item Teilen — das Material in jedwedem Format oder Medium vervielfältigen und weiterverbreiten
		\item Bearbeiten — das Material remixen, verändern und darauf aufbauen
	\end{itemize}
	Unter folgenden Bedingungen:
	\begin{itemize}
		\item Namensnennung - Sie müssen angemessene Urheber- und Rechteangaben machen , einen Link zur Lizenz beifügen und angeben, ob Änderungen vorgenommen wurden. Diese Angaben dürfen in jeder angemessenen Art und Weise gemacht werden, allerdings nicht so, dass der Eindruck entsteht, der Lizenzgeber unterstütze gerade Sie oder Ihre Nutzung besonders.
		\item Nicht kommerziell - Sie dürfen das Material nicht für kommerzielle Zwecke nutzen.
		\item Weitergabe unter gleichen Bedingungen - Wenn Sie das Material remixen, verändern oder anderweitig direkt darauf aufbauen, dürfen Sie Ihre Beiträge nur unter derselben Lizenz wie das Original verbreiten.
	\end{itemize}
	\newpage
	\tableofcontents
	\thispagestyle{empty}
	\newpage
	%\setlength\extrarowheight{10pt} %Horizontales padding für Tabellen
	\def\pics{./pics}
	\rohead{Entity-Relationship-Modell}
	% !TeX root = ./Skript_DB.tex
\cohead{\Large\textbf{Grundlagen ERM}}
\section[Enity-Relationship-Modell]{Enity-Relationship-Modell}
\subsection[Datenbanken]{Datenbanken - Einführung}
Datenbanken enthalten, wie der Name schon sagt, große Mengen an Daten, z.B. eine Firma, die ihre Kunden mit Anschrift und die zugehörigen Bestellungen speichern muss. Das Verwalten und Durchsuchen von großen Mengen an Daten ist nicht trivial, z.B. würde Excel schnell an seine Grenzen stoßen, wenn z.B. eine Firma ihre Produkte, Kunden, Bestellungen, usw. speichern will. Datenbanken wurden genau zu diesem Zweck, dem Speichern und Bearbeiten von großen Mengen an Daten entwickelt.

Die Daten sollen übersichtlich gespeichert werden und so, dass man sie bearbeiten und durchsuchen kann. Dafür benötigt man eine Struktur. Stellen wir uns vor, eine Firma würde alle Daten, also Bestellungen, Kundendaten, Produktbeschreibungen, Preise, Rechnungen, Daten zu Angestellten, usw. einfach ausdrucken und in einen riesigen Container zusammen werfen. Die Daten wären zwar vorhanden, aber sucht man nun nach der Bestellung von John Wick, weil er sich beschwert, einen Toaster statt eines Eierkochers bekommen zu haben, so würde dies exorbitant viel Zeit in Anspruch nehmen. Würde man jedoch alle Bestellungen zusammen in einem Ordner sammeln, wäre die Suche deutlich einfacher. Die Daten sind dann strukturiert und damit übersichtlicher und einfacher zu bearbeiten. Etwas Ähnliches macht man auch mit den Daten in einer Datenbank. Sehr häufig wird das Enity-Relationship-Modell (ERM) verwendet, um die Daten zu strukturieren.
\subsection[Grundlagen]{Enity-Relationship-Modell}
Das Enity-Relationship-Modell (ERM) dient dazu die Struktur von Daten darzustellen, z.B., dass ein Kunde über einen Namen, Vornamen und eine Adresse verfügt. Wie genau ein bestimmter Kunde heißt oder wo er wohnt spielt für das ERM keine Rolle. Dazu wird eine Grafik, das ER-Diagramm angefertigt sowie eine Beschreibung der Elemente dieser Grafik. In unseren Beispielen werden die Elemente selbsterklärend sein und wir werden uns die Beschreibung sparen (Im obigen Beispiel ist klar, was Name, Vorname und Adresse des Kunden sind. Es ist keine zusätzliche Beschreibung notwendig).

Für die grafische Darstellungen werden wir die \texttt{Chen-Notation} verwenden. Die wesentlichen Elemente eines ERMs sind:
\begin{tcolorbox}[title=Entitätstypen und Entitäten]
	Darstellung eines meistens in der Realität vorhandenen Objekts auf der Datenbank, z.B. Kunde, Schüler oder Rechnung. Der Entitätstyp ist die abstrakte Darstellung, z.B. Schüler, während eine Entität eine konkrete Ausprägung, also ein Beispiel ist. So wäre der Schüler Momen Subotic eine Entität in der Datenbank.
	Weitere Beispiele für Entitäten sind:
	\begin{itemize}
		\item Individuen: Person Heinrich Müller, Firma GehtganzGut, Kunde Maria Meyer
		\item Konkreter Gegenstand: Raum A-308, Abteilung Lohn\&Gehalt, Wohnort Berlin
		\item Ereignis: Buchung, Mahnung, Vermietung
		\item Abstraktes: Unterricht, Klasse, Zahlungsart, Tagesplan
	\end{itemize}
	Eine Entität ist immer Mitglied einer Gruppe, des Entitätstyps. Diese kategorisiert also Entitäten mit gleichen Eigenschaften. So sind z.B. die Schülerin Christine Adler und der Schüler Christopher Jäger konkrete individuell identifizierbare Objekte, zu denen Informationen gespeichert werden. Da sie aber die gleichen Eigenschaften haben, gehören sie zum Entitätstyp Schüler. Welche Objekte so wichtig sind, dass sie als Entitätstyp in das Datenbankmodell aufgenommen werden sollen, muss sich an den funktionellen und informatorischen Zusammenhängen der zu speichernden Daten orientieren.
	Der Entitätstyp, also die Menge der Entitäten, wird in der grafischen Darstellung des ER-Modells, dem ER-Diagramm, als Rechteck dargestellt und die Bezeichnung eines Entitätstyps ist immer ein Substantiv.
\end{tcolorbox}
\begin{tcolorbox}[title=Beziehungen]
	Verknüpfungen von Entitäten, z.B. ist eine Rechnung immer einem bestimmten Kunden zugeordnet. Oft kann ein Entitätstyp Beziehungen zu vielen anderen Entitätstypen haben. So könnte die Rechnung nicht nur mit einem Kunden, sondern auch mit dem Entitätstyp Produkt verknüpft sein, der die bestellten Waren angibt.

	Durch Beziehungen werden die Wechselwirkungen oder Abhängigkeiten von Entitäten ausgedrückt. Beziehungen können ebenfalls Attribute (Eigenschaften) besitzen. Ein Beziehungstyp ist, analog zum Entitätstyp, die Abstraktion gleichartiger Beziehungen. Die Beziehung wird dabei meist durch Verben beschrieben und soll in Beziehungsrichtung einen vollständigen Satz ergeben.

	Beispiele für Beziehungen:
	\begin{itemize}
		\item Kind gehört zu Eltern
		\item Team verfügt über Betreuer
		\item Team besteht aus Teammitglied
	\end{itemize}
	Der Beziehungstyp wird grafisch durch eine Raute dargestellt, die durch zwei Kanten mit den Entitätstypen verbunden ist. In der Raute steht der Name des Beziehungstyps.
\end{tcolorbox}
\begin{tcolorbox}[title=Attribute]
	Attribute, auch als Eigenschaft oder Merkmal bezeichnet, beschreiben die Entitäten näher. Alle Entitäten eines Entitätstyps besitzen dieselben Attribute, jedoch sind die Attributswerte unterschiedlich. Attribute charakterisieren also eine Entität, einen Entitätstyp, eine Beziehung bzw. einen Beziehungstyp.

	Beispiele für Attribute:
	\begin{itemize}
		\item Name, Vorname oder Adresse einer Person
		\item Betrag einer Rechnung oder Bestellung
		\item Klassengröße oder Klassenzimmer einer Klasse
	\end{itemize}
	In der grafischen Darstellung werden Attribute als Ellipsen oder Kreise dargestellt. Diese sind über ungerichtete Kanten mit dem Entitätstyp verbunden.
\end{tcolorbox}
\begin{minipage}{\textwidth}
	\begin{minipage}{0.33\textwidth}
		\centering\includegraphics[width=4cm]{\pics/ERMEntity.png}

		Grafische Darstellung eines Entitätstyps.
	\end{minipage}
	\begin{minipage}{0.33\textwidth}
		\centering\includegraphics[width=4cm]{\pics/ERMRelationship.png}

		Grafische Darstellung einer Beziehung.
	\end{minipage}
	\begin{minipage}{0.33\textwidth}
		\centering\includegraphics[width=4cm]{\pics/ERMAttribute.png}

		Grafische Darstellung eines Attributs.
	\end{minipage}
\end{minipage}
\begin{Exercise}[title={Beantworte folgende Fragen.}, label=ERMFragen1]
	\begin{enumerate}
		\item Worin liegt der Unterschied zwischen Entitäten und Entitätstypen?
		\item Worin liegt der Unterschied zwischen Attributen und Attributswerten?
	\end{enumerate}
\end{Exercise}
%%%%%%%%%%%%%%%%%%%%%%%%%%%%%%%%%%%%%%%%%
\begin{Answer}[ref=ERMFragen1]
	\begin{enumerate}
		\item Worin liegt der Unterschied zwischen Entitäten und Entitätstypen?

		Ein Entitätstyp ist der abstrakte Übergriff, z.B. Schüler, während eine Entität ein konkretes Beispiel ist, z.B. der Schüler Noah Schimek ist eine Entität vom Entitätstyp Schüler.
		\item Worin liegt der Unterschied zwischen Attributen und Attributswerten?

		Auch hier ist das Attribut die abstrakte Eigenschaft, z.B. hat der Entitätstyp Schüler ein Attribut Namen. Den Wert eines Attributs erhält man, wenn man eine bestimmte Entität betrachtet, z.B. hat das Attribut Namen beim obigen Schüler den Wert Noah Schimek.
	\end{enumerate}
\end{Answer}
\begin{Exercise}[title={Erstelle jeweils ein ERM}, label=ERMErstellen1]
	\begin{enumerate}
		\item Ein Fahrradverleih am Bodensee verleiht Damen-, Herren- und Kinderfahrräder. Dabei wird für jedes Fahrrad ein eigener Mietvertrag abgeschlossen. Eine Person kann mehrere Fahrräder mieten. Der Fahrradverleih möchte eine Datenbank aufbauen. Helfen Sie dabei.
		\item Ein befreundeter Autohändler bittet uns beim Aufbau einer Kundendatenbank zu helfen. Zuerst soll diese in einem ERM modelliert werden. Darin erscheinen sollen Kunde, Auto, Karosserietyp und Reifen. Ein Auto gehört dabei zu einem Kunden, ein Kunde kann aber mehrere Autos haben.
		\item Ein DVD-Verleiher betreibt mehrere Filialen (id, strasse, plz), wo es jeweils mehrere Medien (DVDs, BluRays, Spiele) zu leihen gibt. Jeder Kunde kann nur einer Filiale zugeordnet sein. Jeder Kunde kann mehrere Medien ausleihen. Ein Mitarbeiter kann nur in einer Filiale arbeiten.
	\end{enumerate}
\end{Exercise}
%%%%%%%%%%%%%%%%%%%%%%%%%%%%%%%%%%%%%%%%%
\begin{Answer}[ref=ERMErstellen1]
	Anmerkung: Die ERMs sind nur Lösungsvorschläge. Man kann auch zusätzliche Attribute hinzufügen. Zudem sind die ERMs nicht vollständig, wir werden später noch neue Begriffe kennen lernen, die hier noch fehlen.
	\begin{enumerate}
		\item Ein Fahrradverleih am Bodensee verleiht Damen-, Herren- und Kinderfahrräder. Dabei wird für jedes Fahrrad ein eigener Mietvertrag abgeschlossen. Eine Person kann mehrere Fahrräder mieten. Der Fahrradverleih möchte eine Datenbank aufbauen. Helfen Sie dabei.

		\begin{minipage}{0.8\textwidth}
			\centering\includegraphics[width=\textwidth]{\pics/ERMFahrrad.png}
		\end{minipage}

		\item Ein befreundeter Autohändler bittet uns beim Aufbau einer Kundendatenbank zu helfen. Zuerst soll diese in einem ERM modelliert werden. Darin erscheinen sollen Kunde, Auto, Karosserietyp und Reifen. Ein Auto gehört dabei zu einem Kunden, ein Kunde kann aber mehrere Autos haben.

		\begin{minipage}{0.8\textwidth}
			\centering\includegraphics[width=\textwidth]{\pics/ERMAuto.png}
		\end{minipage}

		\item Ein DVD-Verleiher betreibt mehrere Filialen (id, strasse, plz), wo es jeweils mehrere Medien (DVDs, BluRays, Spiele) zu leihen gibt. Jeder Kunde kann nur einer Filiale zugeordnet sein. Jeder Kunde kann mehrere Medien ausleihen. Ein Mitarbeiter kann nur in einer Filiale arbeiten.

		\begin{minipage}{0.8\textwidth}
			\centering\includegraphics[width=\textwidth]{\pics/ERMDVDs.png}
		\end{minipage}
	\end{enumerate}
\end{Answer}

\subsection{Primärschlüssel und Kardinalitäten}
Der Primärschlüssel löst das Problem der Eindeutigkeit. Jede Entität, die auf der Datenbank gespeichert wird, muss eindeutig identifizierbar sein. Speichert z.B. eine Firma die Entität Kunde, so muss jeder Kunde eindeutig identifizierbar sein. Würde man als Attribute nur den Namen und Vornamen anhängen, so könnte man Diego Maradonna aus Bremen nicht von Diego Maradonna aus Stuttgart unterscheiden. Man kann natürlich einfach zusätzliche Attribute hinzufügen, wie z.B. das Geburtsdatum oder die Adresse, bis man sich sicher ist, dass es nicht zu Verwechslungen kommen kann, aber es gibt einen eleganteren Weg. Im Normalfall hängt man eine Nummer an (z.B. die Kundennummer), die für jede Entität eine andere sein muss. So kann man jede Entität eindeutig an Hand der Nummer identifizieren. Diego mit der Kundennummer 44445 ist dann eine andere Person als Diego mit der Kundennummer 85417. Diese Nummer bezeichnet man Primärschlüssel.
\begin{tcolorbox}[title=Primärschlüssel]
	Attribut, das eine Entität eindeutig identifizierbar macht. Im Normalfall eine laufende Nummer, d.h. bei jeder neu hinzukommenden Entität wird die Nummer einfach um eins größer gemacht. Der Primärschlüssel wird im ERM durch Unterstreichen kenntlich gemacht.
\end{tcolorbox}
\begin{tcolorbox}[title=Fremdschlüssel]
	Wird der Primärschlüssel eines Entitätstyps an einen anderen Entitätstyp als Attribut hinzugefügt, so bezeichnet man dieses Attribut als Fremdschlüssel.
\end{tcolorbox}
Die Kardinalität gehört zu Beziehungen und gibt an, wie viele Entitäten jeweils in Beziehung zueinander stehen können. Diesen Angaben schreibt man auf die Kanten zwischen den jeweiligen Entitätstypen und der Beziehung. Man unterscheidet im Wesentlichen folgende Kardinalitäten:

\begin{itemize}
	\item 1:1 Beziehung

	Jede Entität des einen Typs \(E_1\) ist maximal einer Entität des anderen Typs \(E_2\) zugeordnet und umgekehrt, z.B. hat jedes Land genau eine Hauptstadt und jede Hauptstadt liegt in genau einem Land.
	\begin{minipage}{0.8\textwidth}
		\centering\includegraphics[width=\textwidth]{\pics/ERMBeziehungen11.png}
		Grafische Darstellung einer 1:1 Beziehung.
	\end{minipage}
	\item 1:N Beziehung

	Jeder Entität des einen Typs \(E_1\) sind beliebig viele Entitäten des zweiten Typs \(E_2\) zugeordnet. Umgekehrt sind jedoch jeder Entität vom Typ \(E_2\) maximal eine Entität vom Typ \(E_1\) zugeordnet, z.B. gehen mehrere Schüler in eine Klasse, umgekehrt geht aber ein einzelner Schüler in genau eine Klasse.

	\begin{minipage}{0.8\textwidth}
		\centering\includegraphics[width=\textwidth]{\pics/ERMBeziehungen1N.png}

		Grafische Darstellung einer 1:1 Beziehung.
	\end{minipage}
	\item N:M Beziehung

	Jeder Entität des einen Typs \(E_1\) sind beliebig viele Entitäten des zweiten Typs \(E_2\) zugeordnet und umgekehrt, z.B. kann ein Mitarbeiter an mehreren Projekten gleichzeitig arbeiten und umgekehrt können an einem Projekt mehrere Mitarbeiter gleichzeitig arbeiten.

	\begin{minipage}{0.8\textwidth}
		\centering\includegraphics[width=\textwidth]{\pics/ERMBeziehungenNM.png}

		Grafische Darstellung einer 1:1 Beziehung.
	\end{minipage}
\end{itemize}

\subsection{Beispiel eines ERMs einer Firma}
Die Kunden einer Firma können Bestellungen aufgeben, die ein oder mehrere Artikel enthalten. Zu jeder Bestellung erhält der Kunde eine Rechnung. Für die Firma wichtig sind Name und Adresse der Kunden, das Datum der Bestellung, welche Artikel enthalten sind und wie teuer diese sind sowie der Betrag und das Fälligkeitsdatum der Rechnung:

\begin{minipage}{\textwidth}
	\centering\includegraphics[width=\textwidth]{\pics/ERMFirmaEinfach.png}
\end{minipage}

Beispiel für ein ERM. Die Primärschlüssel sind jeweils unterstrichen. Die Kardinalitäten der Beziehungen ergeben sich aus folgenden Überlegungen:
\begin{itemize}
	\item Kunde zu Bestellung eine 1:N Beziehung. Ein Kunde kann mehrere Bestellungen aufgeben, aber jede Bestellung ist genau einem Kunden zugeordnet.
	\item Rechnung zu Bestellung eine 1:1 Beziehung. Hinter jeder Rechnung verbirgt sich genau eine Bestellung und zu jeder Bestellung wird genau eine Rechnung erstellt.
	\item Bestellung zu Artikel eine N:M Beziehung. In jeder Bestellung können mehrere (M verschiedene) Artikel vorkommen und umgekehrt kann ein Artikel in verschiedenen Bestellungen (N verschiedene) vorkommen.
\end{itemize}
\begin{Exercise}[title=Vervollständige die ERMs aus Aufgabe \ref{ERMErstellen1}. Jeder Entitätstyp muss einen Primärschlüssel haben und ergänze die Kardinalitäten., label=ERMErstellen2]
	\phantom{ }
\end{Exercise}
%%%%%%%%%%%%%%%%%%%%%%%%%%%%%%%%%%%%%%%%%%
\begin{Answer}[ref=ERMErstellen2]
	\begin{enumerate}
		\item Ein Fahrradverleih am Bodensee verleiht Damen-, Herren- und Kinderfahrräder. Dabei wird für jedes Fahrrad ein eigener Mietvertrag abgeschlossen. Eine Person kann mehrere Fahrräder mieten. Der Fahrradverleih möchte eine Datenbank aufbauen. Helfen Sie dabei.

		\begin{minipage}{0.8\textwidth}
			\centering\includegraphics[width=\textwidth]{\pics/ERMFahrradvoll.png}
		\end{minipage}
		\item Ein befreundeter Autohändler bittet uns beim Aufbau einer Kundendatenbank zu helfen. Zuerst soll diese in einem ERM modelliert werden. Darin erscheinen sollen Kunde, Auto, Karosserietyp und Reifen. Ein Auto gehört dabei zu einem Kunden, ein Kunde kann aber mehrere Autos haben.

		\begin{minipage}{0.8\textwidth}
			\centering\includegraphics[width=\textwidth]{\pics/ERMAutovoll.png}
		\end{minipage}
		\item Ein DVD-Verleiher betreibt mehrere Filialen (id, strasse, plz), wo es jeweils mehrere Medien (DVDs, BluRays, Spiele) zu leihen gibt. Jeder Kunde kann nur einer Filiale zugeordnet sein. Jeder Kunde kann mehrere Medien ausleihen. Ein Mitarbeiter kann nur in einer Filiale arbeiten.

		\begin{minipage}{0.8\textwidth}
			\centering\includegraphics[width=\textwidth]{\pics/ERMDVDsvoll.png}
		\end{minipage}
	\end{enumerate}
\end{Answer}

\begin{Exercise}[title=Erstelle ein ERM., label=ERMErstellen3]
	In der Oberstufe eines Gymnasiums wird nicht mehr in Klassen, sondern in Kursen unterrichtet. Sie erhalten von der Schulleitung den Auftrag, eine Kursverwaltung mittels des Entity-Relationship-Modells zu modellieren. Mit Hilfe dieser Kursverwaltung soll festgehalten werden, welche Schüler welche Kurse besuchen. Als Schülerdaten soll neben dem Vornamen und Nachnamen der Schüler auch die individuelle Schülernummer, das Geburtsdatum, Geschlecht sowie die Postadresse festgehalten	werden. Jeder Kurs hat eine eigene Kursnummer. Außerdem sind der Kurstyp (5stündig / 2stündig), das Fach (z.B. D, M, E, …) und die Jahrgangsstufe (K1 / K2) zu speichern. In dem neuen System sollen auch die Fehlstunden und Kursnoten jedes Schülers dokumentiert werden. Jeder Kurs ist einem Lehrer zugeordnet. Als lehrerspezifische Daten sollen dessen Vor- und Nachname, das Kürzel und seine Fächer (max. 2) mit in das Kursverwaltungsprogramm aufgenommen werden.
\end{Exercise}
\begin{Answer}[ref=ERMErstellen3]
	In der Oberstufe eines Gymnasiums wird nicht mehr in Klassen, sondern in Kursen unterrichtet. Sie erhalten von der Schulleitung den Auftrag, eine Kursverwaltung mittels des Entity-Relationship-Modells zu modellieren. Mit Hilfe dieser Kursverwaltung soll festgehalten werden, welche Schüler welche Kurse besuchen. Als Schülerdaten soll neben dem Vornamen und Nachnamen der Schüler auch die individuelle Schülernummer, das Geburtsdatum, Geschlecht sowie die Postadresse festgehalten werden. Jeder Kurs hat eine eigene Kursnummer. Außerdem sind der Kurstyp (5stündig / 2stündig), das Fach (z.B. D, M, E, …) und die Jahrgangsstufe (K1 / K2) zu speichern. In dem neuen System sollen auch die Fehlstunden und Kursnoten jedes Schülers dokumentiert werden. Jeder Kurs ist einem Lehrer zugeordnet. Als lehrerspezifische Daten sollen dessen Vor- und Nachname, das Kürzel und seine Fächer (max. 2) mit in das Kursverwaltungsprogramm aufgenommen werden.

	\begin{minipage}{\textwidth}
		\centering\includegraphics[width=\textwidth]{\pics/ERMOberstufe.png}
	\end{minipage}
\end{Answer}

	\newpage
	% !TeX root = ../Skript_DB.tex
\cohead{\Large\textbf{Darstellung auf der DB}}
\subsection[Darstellung auf der DB]{Darstellung auf der Datenbank}
Einerseits ist die Darstellung eines ERMs auf der Datenbank simpel. Jeder Entitätstyp und jede Beziehung kann als Tabelle dargestellt werden. Andererseits ergeben sich aus der Darstellung bestimmte Regeln zum Erstellen eines guten ERMs, die wir unter dem Kapitel Normalisierung besprechen werden.

\begin{tcolorbox}[title=Entitäten/-stypen und Beziehungen auf der Datenbank]
	Entitäten bzw. Entitätstypen und Beziehungen werden als Tabellen auf der Datenbank angelegt. Eine Tabelle besteht aus einer Überschrift, die den Entitätstyp bzw. die Beziehung angibt, den Spaltenüberschriften, die die Attribute darstellen und den Zeilen mit konkreten Werten. Die Werte einzelner Zellen stehen dabei für die Attributswerte während eine ganze Zeile eine Entität repräsentiert.
\end{tcolorbox}
Beispiel für die Entitätstypen Schüler, Lehrer und Abteilungen. Für die Namen auf der Datenbank wird im Skript folgende Notation verwendet: \lstinline!schueler!

\medskip

\begin{tabular}{llll}
	\multicolumn{4}{c}{\lstinline!schueler!}\\
	\hline
	\underline{\lstinline!schuelerNR!}&\lstinline!name!&\lstinline!klasse!&\lstinline!klassenlehrer!\\
	\hline
	15&Waylon Smithers&BK22&Krabappel\\
	24&Moe Szyslak&BK13&Skinner\\
	102&Homer Simpson&BK22&Krabappel\\
	9&Apu Nahasapeemapetilon&WG32&Krabappel\\
	47&Carl Carlson&BK12&Skinner\\
\end{tabular}

\begin{tabular}{lll}
	\multicolumn{3}{c}{\lstinline!abteilungen!}\\
	\hline
	\underline{\lstinline!abteilungNR!}&\lstinline!bezeichnung!&\lstinline!abteilungsleiter!\\
	\hline
	1&Berufskolleg&Krabappel\\
	2&Wirtschaftsgymnasium&Hoover\\
\end{tabular}

\begin{tabular}{lll}
	\multicolumn{3}{c}{\lstinline!lehrer!}\\
	\hline
	\underline{\lstinline!lehrerNR!}&\lstinline!name!&\lstinline!kuerzel!\\
	\hline
	24&Krabappel&KRB\\
	30&Skinner&SKI\\
	31&Hoover&HOO\\
\end{tabular}

\medskip

Wie wir später sehen werden, ist das gewählte ERM noch stark verbesserungsfähig. Vorerst wollen wir aber jeden Lehrer genau einer Abteilung zuordnen. Diese Beziehung kann ebenfalls über eine Tabelle dargestellt werden:

\medskip

\begin{tabular}{lll}
	\multicolumn{3}{c}{\lstinline!abteilung_lehrer!}\\
	\hline
	\underline{\lstinline!lfdNR!}&\lstinline!abteilungNR!&\lstinline!lehrerNR!\\
	\hline
	1&1&24\\
	2&1&30\\
	3&2&31\\
\end{tabular}

\medskip

In der Tabelle \lstinline!abteilung_lehrer! könnte man die \lstinline!lfdNR! als Primärschlüssel auch streichen und stattdessen die \lstinline!abteilungNR! und \lstinline!lehrerNR! als zusammengesetzten Primärschlüssel verwenden. Außer dem geringfügig größeren Speicherbedarf spricht aber für unsere Zwecke nichts gegen das Hinzufügen einer laufenden Nummer als Primärschlüssel. Die erste Entität bzw. Zeile sagt nun aus, dass der Abteilung mit der \lstinline!abteilungNR! 1, also dem Berufskolleg, der Lehrer mit der \lstinline!lehrerNR! 24, also Krabappel, angehört. In der zweiten Zeile bzw. Entität steht, dass dem Berufskolleg außerdem auch Skinner angehört. Und die dritte Zeile besagt, dass dem Wirtschaftsgymnasium Hoover angehört.

Beachte, dass der Primärschlüssel auch hier unterstrichen ist und aus Gründen der Übersichtlichkeit immer an erster Stelle steht.
\begin{Exercise}[title=Erstelle zu den ERMs aus Aufgabe \ref{ERMErstellen1} passende Tabellen., label=TabelleErstellen1]

\end{Exercise}
%%%%%%%%%%%%%%%%%%%%%%%%%%%%%%%%%%%%%%%%%%
\begin{Answer}[ref=TabelleErstellen1]
	\begin{enumerate}
		\item Ein Fahrradverleih am Bodensee verleiht Damen-, Herren- und Kinderfahrräder. Dabei wird für jedes Fahrrad ein eigener Mietvertrag abgeschlossen. Eine Person kann mehrere Fahrräder mieten. Der Fahrradverleih möchte eine Datenbank aufbauen. Helfen Sie dabei.

        \medskip

        \begin{minipage}{\linewidth}
            \adjustbox{valign=t}{\begin{minipage}{.7\linewidth}
            \begin{tabular}{lll}
    			\multicolumn{3}{c}{\lstinline!kunden!}\\
    			\hline
    			\underline{\lstinline!kundeNR!}&\lstinline!name!&\lstinline!adresse!\\
    			\hline
    			15&Fathi Russdi&Rotgasse 12, 55555 Bielefeld
    		\end{tabular}\end{minipage}}%
            \adjustbox{valign=t}{\begin{minipage}{.3\linewidth}
    		\begin{tabular}{ll}
    			\multicolumn{2}{c}{\lstinline!fahrraeder!}\\
    			\hline
    			\underline{\lstinline!rahmenNR!}&\lstinline!typ!\\
    			\hline
    			18498451&Rennrad\\
    			89415456&Kinderrad
    		\end{tabular}\end{minipage}}%
        \end{minipage}

		\begin{tabular}{lllll}
			\multicolumn{5}{c}{\lstinline!mietet!}\\
			\hline
			\underline{\lstinline!laufendeNR!}&\lstinline!kundenNR!&\lstinline!rahmenNR!&\lstinline!beginn!&\lstinline!ende!\\
			\hline
			1&15&18498451&01.09.2023&03.09.2023\\
			2&15&89415456&01.09.2023&03.09.2023\\
		\end{tabular}

        \medskip

		\item Ein befreundeter Autohändler bittet uns beim Aufbau einer Kundendatenbank zu helfen. Zuerst soll diese in einem ERM modelliert werden. Darin erscheinen sollen Kunde, Auto, Karosserietyp und Reifen. Ein Auto gehört dabei zu einem Kunden, ein Kunde kann aber mehrere Autos haben.

		\begin{tabular}{lll}
			\multicolumn{3}{c}{\lstinline!kunden!}\\
			\hline
			\underline{\lstinline!kundeNR!}&\lstinline!name!&\lstinline!adresse!\\
			\hline
			15&Fathi Russdi&Rotgasse 12, 55555 Bielefeld\\
		\end{tabular}

        \medskip

		\begin{tabular}{ll}
			\multicolumn{2}{c}{\lstinline!autos!}\\
			\hline
			\underline{\lstinline!fahrzeugID!}&\lstinline!karosserie!\\
			\hline
			189765465451&Kübelwagen\\
		\end{tabular}

        \medskip

		\begin{tabular}{ll}
			\multicolumn{2}{c}{\lstinline!reifen!}\\
			\hline
			\underline{\lstinline!reifenID!}&\lstinline!typ!\\
			\hline
			89454115&Winterreifen\\
		\end{tabular}

        \medskip

		\begin{tabular}{lll}
			\multicolumn{3}{c}{\lstinline!auto_reifen!}\\
			\hline
			\underline{\lstinline!laufendeNR!}&\lstinline!fahrzeugID!&\lstinline!reifenID!\\
			\hline
			4&189765465451&89454115\\
		\end{tabular}

        \medskip

		\begin{tabular}{lllll}
			\multicolumn{5}{c}{\lstinline!auto_kunde!}\\
			\hline
			\underline{\lstinline!laufendeNR!}&\lstinline!fahrzeugID!&\lstinline!kundenNR!&\lstinline!preis!&\lstinline!datum!\\
			\hline
			119&189765465451&15&43000&10.08.2023\\
		\end{tabular}

        \medskip

		\item Ein DVD-Verleiher betreibt mehrere Filialen (id, strasse, plz), wo es jeweils mehrere Medien (DVDs, BluRays, Spiele) zu leihen gibt. Jeder Kunde kann nur einer Filiale zugeordnet sein. Jeder Kunde kann mehrere Medien ausleihen. Ein Mitarbeiter kann nur in einer Filiale arbeiten.

		\begin{tabular}{lll}
			\multicolumn{3}{c}{\lstinline!filialen!}\\
			\hline
			\underline{\lstinline!filialID!}&\lstinline!strasse!&\lstinline!plz!\\
			\hline
			3&Königsbau 4&70174\\
		\end{tabular}

        \medskip

		\begin{tabular}{ll}
			\multicolumn{2}{c}{\lstinline!mitarbeiter!}\\
			\hline
			\underline{\lstinline!mitarbeiterNR!}&\lstinline!name!\\
			\hline
			47&Agent 47\\
		\end{tabular}

        \medskip

		\begin{tabular}{llll}
			\multicolumn{4}{c}{\lstinline!medien!}\\
			\hline
			\underline{\lstinline!medienID!}&\lstinline!bezeichnung!&\lstinline!anzahl!&\lstinline!typ!\\
			\hline
			1002&The matrix&5&DVD\\
		\end{tabular}

        \medskip

		\begin{tabular}{ll}
			\multicolumn{2}{c}{\lstinline!kunden!}\\
			\hline
			\underline{\lstinline!kundenNR!}&\lstinline!name!\\
			\hline
			50&Lucky Luke\\
		\end{tabular}

        \medskip

		\begin{tabular}{lll}
			\multicolumn{3}{c}{\lstinline!filiale_mitarbeiter!}\\
			\hline
			\underline{\lstinline!laufendeNR!}&\lstinline!filialID!&\lstinline!mitarbeiterNR!\\
			\hline
			1&3&47\\
		\end{tabular}

        \medskip

		\begin{tabular}{llll}
			\multicolumn{4}{c}{\lstinline!filiale_medium!}\\
			\hline
			\underline{\lstinline!laufendeNR!}&\lstinline!filialID!&\lstinline!medienID!&\lstinline!anzahl!\\
			\hline
			1&3&1002&2\\
		\end{tabular}

        \medskip

		\begin{tabular}{lllll}
			\multicolumn{5}{c}{\lstinline!kunde_medium!}\\
			\hline
			\underline{\lstinline!laufendeNR!}&\lstinline!kundenNR!&\lstinline!medienID!&\lstinline!beginn!&\lstinline!beginn!\\
			\hline
			99&50&1002&01.01.2023&14.01.2023\\
		\end{tabular}
	\end{enumerate}
\end{Answer}
	\newpage
	% !TeX root = ./Skript_DB.tex
\cohead{\Large\textbf{Anomalien}}
\subsection[Anomalien]{Anomalien}\label{Anomalien}
Von Anomalien spricht man, wenn die Daten auf der Datenbank inkonsistent sind, also fehlerhaft. Wir unterscheiden folgende Anomalien:

\begin{tcolorbox}[title=Anomalien]
	\begin{enumerate}
		\item \textbf{Änderungs-Anomalien}: Diese können auftreten, wenn Attributswerte an mehreren Stellen geändert werden sollen, jedoch nicht alle Stellen geändert werden.
		\item \textbf{Einfüge-Anomalien}: Diese können auftreten, wenn das Einfügen eines Attributswertes zum zwingenden Einfügen weiterer Attributswerte führt.
		\item \textbf{Lösch-Anomalien}: Diese können auftreten, wenn das Löschen einer Entität das ungewollte Löschen wichtiger Infos (Attributswerte) mit sich bringt.
	\end{enumerate}
\end{tcolorbox}
Zudem entspricht es dem Best Practice Redundanzen (das Speichern der gleichen Information an mehreren Stellen in der Datenbank) zu vermeiden. Betrachten wir folgendes Beispiel:

Unsere Schüler engagieren sich in unterschiedlichen Schulprojekten. Der Verbindungslehrer Herr KeinDBProfi hat zu Verwaltungszwecken folgende Tabelle angelegt:
\begin{tabular}{llllllll}
	\multicolumn{8}{c}{\lstinline!projektinfos!}\\
	\hline
	\underline{\lstinline!schNR!}&\lstinline!name!&\lstinline!vorname!&\lstinline!klasse!&\lstinline!klassenbez!&\lstinline!projektNR!&\lstinline!probez!&\lstinline!prostd!\\
	\hline
	1 &
	Müller &
	Marius &
	BK22 &
	Kaufm. BK2&
	1 &
	Homepage &
	30 \\
	2 &
	Kryof  &
	Yuri &
	BK14 &
	Kaufm. BK1&
	2 &
	Foyergestaltung &
	25 \\
	3 &
	Abadi &
	Ali &
	BK14 &
	Kaufm. BK1&
	1,2 &
	Homepage,&
	10,\\
	&&&&&&Foyergestaltung&15\\
	4 &
	Sanbei &
	Sarah &
	BK22 &
	Kaufm. BK2 &
	1,3 &
	Homepage,&
	15,  \\
	&&&&&&Schulfest&35\\
\end{tabular}

Nun will man folgende Änderungen vornehmen:
\begin{enumerate}
	\item \textcolor{red}{Da es sich um die ÜFA-Homepage handelt, soll die Projekt-Bezeichnung entsprechend geändert werden.}
	\item \textcolor{blue}{Das neue Projekt \texttt{Abschlussfeier} soll in die Tabelle aufgenommen werden.}
	\item \textcolor{ForestGreen}{Die Schüler der BK22 machen ihren Abschluss und verlassen die Schule}
\end{enumerate}
Die geänderte Tabelle sieht nun wie folgt aus:
\begin{tabular}{llllllll}
	\multicolumn{8}{c}{\lstinline!projektinfos!}\\
	\hline
	\underline{\lstinline!schNR!}&\lstinline!name!&\lstinline!vorname!&\lstinline!klasse!&\lstinline!klassenbez!&\lstinline!projektNR!&\lstinline!probez!&\lstinline!prostd!\\
	\hline
	\textcolor{ForestGreen}{\sout{1}} &
	\textcolor{ForestGreen}{\sout{Müller}} &
	\textcolor{ForestGreen}{\sout{Marius}} &
	\textcolor{ForestGreen}{\sout{BK22}} &
	\textcolor{ForestGreen}{\sout{Kaufm. BK2}}&
	\textcolor{ForestGreen}{\sout{1}} &
	\textcolor{red}{\sout{ÜFA-Homepage}} &
	\textcolor{ForestGreen}{\sout{30}} \\
	2 &
	Kryof  &
	Yuri &
	BK14 &
	Kaufm. BK1&
	2 &
	Foyergestaltung &
	25 \\
	3 &
	Abadi &
	Ali &
	BK14 &
	Kaufm. BK1&
	1,2 &
	Homepage,&
	10,\\
	&&&&&&Foyergestaltung&15\\
	\textcolor{ForestGreen}{\sout{4}}&
	\textcolor{ForestGreen}{\sout{Sanbei}}&
	\textcolor{ForestGreen}{\sout{Sarah}}&
	\textcolor{ForestGreen}{\sout{BK22}}&
	\textcolor{ForestGreen}{\sout{Kaufm. BK2}}&
	\textcolor{ForestGreen}{\sout{1,3}}&
	\textcolor{red}{\sout{ÜFA-Homepage,}}&
	\textcolor{ForestGreen}{\sout{15}}\\
	&&&&&&\textcolor{ForestGreen}{\sout{Schulfest}}&\textcolor{ForestGreen}{\sout{35}}\\
	\textcolor{blue}{?}&
	&
	&
	&
	&
	\textcolor{blue}{4} &
	\textcolor{blue}{Abschlussfeier}&
	\\
\end{tabular}\newpage
Es sind drei verschiedene Anomlien aufgetreten
\begin{enumerate}
	\item \textcolor{red}{Die Projektbezeichnung wurde nicht an allen Stellen von Homepage zu ÜFA-Homepage geändert. Es ist eine Änderungs-Anomalie aufgetreten.}
	\item \textcolor{blue}{Das Einfügen des Projekts Abschlussfeier hat zu einer Einfüge-Anomlie geführt. Da noch kein Schüler an dem Projekt arbeitet, muss der Primärschlüssel \lstinline!schNR! leer bleiben, was verboten ist. Die Entität bzw. Zeile wäre dann nicht mehr an Hand des Primärschlüssels eindeutig identifizierbar.}
	\item \textcolor{ForestGreen}{Es ist eine Lösch-Anomalie aufgetreten. Löscht man die Schüler aus der BK22 aus der Tabelle, so wird auch das Projekt \texttt{Schulfest} gelöscht.}
\end{enumerate}
Zudem verfügt die Tabelle über Redundanzen, z.B. werden die Klassenbezeichnungen und Projektbezeichnungen mehrfach gespeichert.
	\newpage
	\subsection[Normalisieren]{Normalisieren von Datenbanken}
Um Anomalien sowie Redundanzen möglichst zu vermeiden und die Datenbank einfach zu verwalten, muss diese normalisiert werden:
\begin{tcolorbox}[title=Erste Normalform]
	Eine Tabelle liegt in der ersten Normalform vor, wenn jeder Attributswert atomar vorliegt, d.h. jeder Wert ist nicht (sinnvoll) weiter zerlegbar. Zudem muss jeder Entitätstyp über einen Primärschlüssel verfügen (Ob als einzelnes Schlüsselattribut oder als Kombination mehrerer Attribute, ist zweitrangig).
\end{tcolorbox}
Betrachten wir folgenden Sachverhalt. Ein DVD-Verleih legt folgende Tabelle an. Der Primärschlüssel besteht hier aus der \underline{\lstinline!kNR!} und \underline{\lstinline!filmID!}.
\begin{tabular}{lllllll}
	\multicolumn{7}{c}{\lstinline!dvdVerleih!}\\
	\hline
	\underline{\lstinline!kNR!}&\underline{\lstinline!filmID!}&\lstinline!name!&\lstinline!plz!&\lstinline!ort!&\lstinline!filmname!&\lstinline!ausg!\\
	\hline
	5&1002&Keanu Reeves&70180&Stuttgart&Rambo&01.02.2023\\
	5&1003&Keanu Reeves&70180&Stuttgart&Rambo2&06.02.2023\\
	7&2018&Will Smith&72070&Tübingen&LotR&04.02.2023\\
\end{tabular}\\
Die Tabelle liegt nicht in der ersten Hauptform vor, da man den Namen noch in Vor- und Nachname aufteilen kann. Der Vorteil ist, dass man die Tabelle dann leichter nach dem Vor- oder Nachnamen sortieren bzw. durchsuchen kann:
\begin{tabular}{llllllll}
	\multicolumn{8}{c}{\lstinline!dvdVerleih!}\\
	\hline
	\underline{\lstinline!kNR!}&\underline{\lstinline!filmID!}&\lstinline!voname!&\lstinline!nachname!&\lstinline!plz!&\lstinline!ort!&\lstinline!filmname!&\lstinline!ausg!\\
	\hline
	5&1002&Keanu&Reeves&70180&Stuttgart&Rambo&01.02.2023\\
	5&1003&Keanu&Reeves&70180&Stuttgart&Rambo2&06.02.2023\\
	7&2018&Will&Smith&72070&Tübingen&LotR&04.02.2023\\
\end{tabular}\\
\begin{tcolorbox}[title=Zweite Normalform]
	Eine Tabelle liegt in der zweiten Normalform vor, wenn sie in der ersten Normalform ist und zusätzlich keine Attribute enthält, die bereits von einem Teil eines Schlüsselkandidaten eindeutig bestimmt werden. Somit muss jedes Nichtschlüsselattribut voll funktional (d.h. ausschließlich) abhängig vom Primärschlüssel sein.
\end{tcolorbox}
In diesem Fall ist der \lstinline!filmname! nicht von \underline{\lstinline!kNR!} und \underline{\lstinline!filmID!} abhängig, sondern nur von einem Teil des Primärschlüssels, nämlich der \underline{\lstinline!filmID!}. Selbiges gilt für \lstinline!vorname! und \lstinline!nachname!, die nur von \lstinline!kNR! abhängig sind. Um dieses Problem zu lösen, legen wir zwei zusätzliche Tabelle an:\\
\begin{minipage}{\textwidth}
	\begin{minipage}{0.3\textwidth}
		\begin{tabular}{lll}
			\multicolumn{3}{c}{\lstinline!leiht!}\\
			\hline
			\underline{\lstinline!kNR!}&\underline{\lstinline!filmID!}&\lstinline!ausg!\\
			\hline
			5&1002&01.02.2023\\
			5&1003&06.02.2023\\
			7&2018&04.02.2023\\
		\end{tabular}
	\end{minipage}
	\begin{minipage}{0.228\textwidth}
		\begin{tabular}{ll}
			\multicolumn{2}{c}{\lstinline!filme!}\\
			\hline
			\underline{\lstinline!filmID!}&\lstinline!filmname!\\
			\hline
			1002&Rambo\\
			1003&Rambo2\\
			2018&LotR\\
		\end{tabular}
	\end{minipage}
	\begin{minipage}{0.472\textwidth}
		\begin{tabular}{lllll}
			\multicolumn{5}{c}{\lstinline!kunden!}\\
			\hline
			\underline{\lstinline!kNR!}&\lstinline!vorname!&\lstinline!nachname!&\lstinline!plz!&\lstinline!ort!\\
			\hline
			5&Keanu&Reeves&70180&Stuttgart\\
			7&Will&Smith&72070&Tübingen\\
			\phantom{0}&&&&\\
		\end{tabular}
	\end{minipage}
\end{minipage}
\begin{tcolorbox}[title=Dritte Normalform]
	Eine Tabelle liegt in der dritten Normalform vor, wenn sie in der zweiten Normalform ist und zusätzlich keine Attribute enthält, die transitiv abhängig sind, d.h. Attribute, die nicht direkt vom Primärschlüssel abhängen.
\end{tcolorbox}
In diesem Fall ist in der Tabelle \lstinline!kunden! der \lstinline!ort! nicht von der \lstinline!kNR!, sondern von der \lstinline!plz! abhängig. Auch hier schafft eine Aufteilung in zwei Tabellen Abhilfe:
\begin{minipage}{\textwidth}
	\begin{minipage}{0.66\textwidth}
		\begin{tabular}{llll}
			\multicolumn{4}{c}{\lstinline!kunden!}\\
			\hline
			\underline{\lstinline!kNR!}&\lstinline!vorname!&\lstinline!nachname!&\lstinline!plz!\\
			\hline
			5&Keanu&Reeves&70180\\
			7&Will&Smith&72070\\
		\end{tabular}
	\end{minipage}
	\begin{minipage}{0.33\textwidth}
		\begin{tabular}{ll}
			\multicolumn{2}{c}{\lstinline!ort!}\\
			\hline
			\underline{\lstinline!plz!}&\lstinline!ort!\\
			\hline
			70180&Stuttgart\\
			72070&Tübingen\\
		\end{tabular}
	\end{minipage}
\end{minipage}

\begin{Exercise}[title=Normalisiere die Tabelle \lstinline|projektinfos| aus dem Kapitel \ref{Anomalien}., label=Normal0]

\end{Exercise}
\begin{Exercise}[title=Normalisiere folgende Tabelle und markiere die Primärschlüssel in deinem Ergebnis., label=Normal1]
	Ein Bauunternehmer hat die folgende Tabelle erstellt, in der Daten über Bauaufträge und Daten zu den beteiligten Mitarbeitern gespeichert sind:\\
	\begin{tabular}{lllllllll}
		\multicolumn{9}{c}{\lstinline!bauunternehmer!}\\
		\hline
		\lstinline!aNR!&\lstinline!auftrag!&\lstinline!baust!&\lstinline!pNR!&\lstinline!mitarbeiter!&\lstinline!plz!&\lstinline!wohnort!&\lstinline!kkasse!&\lstinline!kkbeitrag!\\
		\hline
		A1&Garage&Stuttgart&13&Cem Özdemir&72070&Tübingen&AOKBW&16,2\\
		A2&Haus&Esslingen&13&Cem Özdemir&72070&Tübingen&AOKBW&16,2\\
		A2&Haus&Esslingen&17&Christian Lindner&70794&Filderstadt&DAK&16,3\\
	\end{tabular}
\end{Exercise}
\begin{Exercise}[title=Normalisiere folgende Tabelle und markiere die Primärschlüssel in deinem Ergebnis., label=Normal2]
	Eine Firma legt ihre Bestellungen wie folgt in einer Datenbank ab. Normalisiere die Tabelle:\\
	\begin{tabular}{lllll}
		\multicolumn{5}{c}{\lstinline!bestellungen!}\\
		\hline
		\underline{\lstinline!kundeNR!}&\lstinline!name!&\lstinline!artNR!&\lstinline!artBez!&\lstinline!anzahl!\\
		\hline
		\multirow{3}{*}{5001}&\multirow{3}{*}{Volker Finke e.K.}&8001&Schraubendreher 5mm&10\\
		&&8005&Schraubendreher 8mm&15\\
		&&8007&Schraubendreher-Set&15\\
		\hline
		\multirow{3}{*}{5004}&\multirow{3}{*}{Hubert Hase GmbH}&8001&Schraubendreher 5mm&20\\
		&&8006&Schraubendreher 10mm&20\\
		&&8007&Schraubendreher-Set&10\\
		\hline
		5007&Rudi Rüssel KG&8007&Schraubendreher-Set&25\\
	\end{tabular}
\end{Exercise}
%%%%%%%%%%%%%%%%%%%%%%%%%%%%%%%%%%%%%%%%%%
\begin{Answer}[ref=Normal0]
	\begin{minipage}[t]{\textwidth}
		\begin{minipage}{0.33\textwidth}
			\begin{tabular}{lll}
				\multicolumn{3}{c}{\lstinline!schueler!}\\
				\hline
				\underline{\lstinline!schNR!}&\lstinline!name!&\lstinline!vorname!\\
				\hline
				1&Käfer&Karl\\
				2&Witzig&Willi\\
				3&Sonne&Susi\\
				4&Mond&Moni\\
			\end{tabular}
		\end{minipage}
		\begin{minipage}[t]{0.33\textwidth}
			\begin{tabular}{ll}
				\multicolumn{2}{c}{\lstinline!projekte!}\\
				\hline
				\underline{\lstinline!projektNR!}&\lstinline!probez!\\
				\hline
				1&Homepage\\
				2&Foyergestaltung\\
				3&Schulfest\\
				\phantom{text}&\\
			\end{tabular}
		\end{minipage}
		\begin{minipage}[t]{0.33\textwidth}
			\begin{tabular}{lll}
				\multicolumn{3}{c}{\lstinline!klasse!}\\
				\hline
				\underline{\lstinline!klasseNR!}&\lstinline!klasse!&\lstinline!klassenbez!\\
				\hline
				1&BK14&Kaufm. BK1\\
				2&BK22&Kaufm. BK2\\
				\phantom{text}&&\\
				\phantom{text}&&\\
			\end{tabular}
		\end{minipage}
	\end{minipage}
	\begin{minipage}{\textwidth}
		\begin{minipage}{0.33\textwidth}
			\begin{tabular}{ll}
				\multicolumn{2}{c}{\lstinline!klassenzug!}\\
				\hline
				\underline{\lstinline!schNR!}&\lstinline!klasseNR!\\
				\hline
				1&2\\
				2&1\\
				3&1\\
				4&2\\
				\phantom{text}&\\
				\phantom{text}&\\
			\end{tabular}
		\end{minipage}
		\begin{minipage}{0.66\textwidth}
			\begin{tabular}{lll}
				\multicolumn{3}{c}{\lstinline!projektteilnahme!}\\
				\hline
				\underline{\lstinline!schNR!}&\underline{\lstinline!projektNR!}&\lstinline!prostd!\\
				\hline
				1&1&30\\
				2&2&25\\
				3&1&10\\
				3&2&15\\
				4&1&15\\
				4&3&35\\
			\end{tabular}
		\end{minipage}
	\end{minipage}
\end{Answer}

\begin{Answer}[ref=Normal1]
	\begin{minipage}{\textwidth}
		\begin{minipage}{0.33\textwidth}
			\begin{tabular}{lll}
				\multicolumn{3}{c}{\lstinline!auftrag!}\\
				\hline
				\underline{\lstinline!aNR!}&\lstinline!auftrag!&\lstinline!baust!\\
				\hline
				A1&Garage&Stuttgart\\
				A2&Haus&Esslingen\\
			\end{tabular}
		\end{minipage}
		\begin{minipage}{0.66\textwidth}
			\begin{tabular}{lllll}
				\multicolumn{5}{c}{\lstinline!personal!}\\
				\hline
				\underline{\lstinline!pNR!}&\lstinline!vorname!&\lstinline!nachname!&\lstinline!plz!&\lstinline!kkasse!\\
				\hline
				13&Cem&Özdemier&72070&AOKBW\\
				17&Christian&Lindner&70794&DAK\\
			\end{tabular}
		\end{minipage}
	\end{minipage}
	\begin{minipage}{\textwidth}
		\begin{minipage}{0.33\textwidth}
			\begin{tabular}{ll}
				\multicolumn{2}{c}{\lstinline!ort!}\\
				\hline
				\underline{\lstinline!plz!}&\lstinline!ort!\\
				\hline
				70794&Filderstadt\\
				72070&Tübingen\\
			\end{tabular}
		\end{minipage}
		\begin{minipage}{0.66\textwidth}
			\begin{tabular}{ll}
				\multicolumn{2}{c}{\lstinline!krankenkasse!}\\
				\hline
				\underline{\lstinline!kkasse!}&\lstinline!kkbeitrag!\\
				\hline
				AOKBW&16,2\\
				DAK&16,3\\
			\end{tabular}
		\end{minipage}
	\end{minipage}
\end{Answer}
\begin{Answer}[ref=Normal2]
	\begin{minipage}{\textwidth}
		\begin{minipage}{0.35\textwidth}
			\begin{tabular}{ll}
				\multicolumn{2}{c}{\lstinline!kunden!}\\
				\hline
				\underline{\lstinline!kundeNR!}&\lstinline!name!\\
				\hline
				5001&Volker Finke e.K.\\
				5004&Hubert Hase GmbH\\
				5007&Rudi Rüssel KG\\
				\vphantom{0}&\\
				\vphantom{0}&\\
				\vphantom{0}&\\
				\vphantom{0}&\\
			\end{tabular}
		\end{minipage}
		\begin{minipage}{0.35\textwidth}
			\begin{tabular}{ll}
				\multicolumn{2}{c}{\lstinline!artikel!}\\
				\hline
				\underline{\lstinline!artNR!}&\lstinline!artBez!\\
				\hline
				8001&Schraubendreher 5mm\\
				8005&Schraubendreher 8mm\\
				8007&Schraubendreher-Set\\
				\vphantom{0}&\\
				\vphantom{0}&\\
				\vphantom{0}&\\
				\vphantom{0}&\\
			\end{tabular}
		\end{minipage}
		\begin{minipage}{0.2\textwidth}
			\begin{tabular}{lll}
				\multicolumn{3}{c}{\lstinline!bestellung!}\\
				\hline
				\underline{\lstinline!kundeNR!}&\underline{\lstinline!artNR!}&\lstinline!anzahl!\\
				\hline
				5001&8001&10\\
				5001&8005&15\\
				5001&8007&15\\
				5004&8001&20\\
				5004&8006&20\\
				5004&8007&10\\
				5007&8007&25\\
			\end{tabular}
		\end{minipage}
	\end{minipage}
\end{Answer}
	\newpage
	% !TeX root = ../Skript_DB.tex
\cohead{\Large\textbf{Optimieren}}
\subsection[Optimierung]{Optimierung von Datenbanken}
Bisher wurde jeder Entitätstyp und jede Beziehung als eigene Tabelle auf der Datenbank dargestellt. Für alle Beziehungen, die von der Kardinalität 1:X oder X:1 sind, kann man die Informationen der Beziehungstabelle in die Tabelle eines der beiden Entitätstypen verschieben.
\begin{tcolorbox}[title=Optimierung]
	Für 1:X oder X:1 Beziehungen kann die Beziehungstabelle in einer der Entitätstyp-Tabellen integriert werden, indem man den Primärschlüssel des einen Entitätstyps in die Tabelle des anderen Entitätstyps als Fremdschlüssel hinzufügt.
\end{tcolorbox}
Betrachten wir folgendes Beispiel:
\begin{minipage}{\textwidth}
	\centering\includegraphics[width=\textwidth]{\pics/Optimierung.png}
\end{minipage}

Bildet man jeden Entitätstyp und jede Beziehung als eine eigene Tabelle ab, so erhält man z.B.:
\begin{minipage}{\textwidth}
	\begin{minipage}{0.5\textwidth}
		\begin{tabular}{lll}
			\multicolumn{3}{c}{\lstinline!schueler!}\\
			\hline
			\underline{\lstinline!schuelerNR!}&\lstinline!vorname!&\lstinline!nachname!\\
			\hline
			105&Max&Verstappen\\
			106&Charles&Leclerc\\
			110&Lewis&Hamilton\\
		\end{tabular}
	\end{minipage}%
	\begin{minipage}{0.5\textwidth}
		\begin{tabular}{ll}
			\multicolumn{2}{c}{\lstinline!klasse!}\\
			\hline
			\underline{\lstinline!klasseNR!}&\lstinline!bezeichnung!\\
			\hline
			1&BK22\\
			2&BK13\\
		\end{tabular}
	\end{minipage}%
\end{minipage}
\begin{minipage}{0.3\textwidth}
	\begin{tabular}{lll}
		\multicolumn{3}{c}{\lstinline!lehrer!}\\
		\hline
		\underline{\lstinline!lehrerNR!}&\lstinline!vorname!&\lstinline!nachname!\\
		\hline
		12&Christian&Horner\\
		13&Frederic&Vasseur\\
	\end{tabular}
\end{minipage}
\begin{minipage}{\textwidth}
	\begin{minipage}{0.5\textwidth}
		\begin{tabular}{ll}
			\multicolumn{2}{c}{\lstinline!schueler_klasse!}\\
			\hline
			\underline{\lstinline!schuelerNR!}&\underline{\lstinline!klasseNR!}\\
			\hline
			105&1\\
			106&1\\
			110&2\\
		\end{tabular}
	\end{minipage}%
	\begin{minipage}{0.5\textwidth}
		\begin{tabular}{ll}
			\multicolumn{2}{c}{\lstinline!lehrer_klasse!}\\
			\hline
			\underline{\lstinline!klasseNR!}&\underline{\lstinline!lehrerNR!}\\
			\hline
			1&12\\
			1&13\\
			2&12\\
			2&13\\
		\end{tabular}
	\end{minipage}%
\end{minipage}
Die Tabelle \lstinline!schueler_klasse! beschreibt die N:1 Beziehung zwischen den Entitätstypen \lstinline!schueler! und \lstinline!klasse!. Da jedem Schüler genau eine Klasse zugeordnet wird, kann man diese Tabelle einsparen, indem man in der Tabelle Schüler als zusätzliches Attribut den Fremdschlüssel \lstinline!klasseNR! einfügt, die direkt die Klasse angibt, in die der Schüler geht:
\begin{minipage}{\textwidth}
	\begin{minipage}{0.6\textwidth}
		\begin{tabular}{llll}
			\multicolumn{4}{c}{\lstinline!schueler!}\\
			\hline
			\underline{\lstinline!schuelerNR!}&\lstinline!vorname!&\lstinline!nachname!&\lstinline!klasseNR!\\
			\hline
			105&Max&Verstappen&1\\
			106&Charles&Leclerc&1\\
			110&Lewis&Hamilton&2\\
		\end{tabular}
	\end{minipage}%
	\begin{minipage}{0.4\textwidth}
		\begin{tabular}{ll}
			\multicolumn{2}{c}{\lstinline!klasse!}\\
			\hline
			\underline{\lstinline!klasseNR!}&\lstinline!bezeichnung!\\
			\hline
			1&BK22\\
			2&BK13\\
		\end{tabular}
	\end{minipage}%
\end{minipage}
\begin{minipage}{\textwidth}
	\begin{minipage}{0.5\textwidth}
		\begin{tabular}{lll}
			\multicolumn{3}{c}{\lstinline!lehrer!}\\
			\hline
			\underline{\lstinline!lehrerNR!}&\lstinline!vorname!&\lstinline!nachname!\\
			\hline
			12&Christian&Horner\\
			13&Frederic&Vasseur\\
		\end{tabular}
	\end{minipage}%
	\begin{minipage}{0.5\textwidth}
		\begin{tabular}{ll}
			\multicolumn{2}{c}{\lstinline!lehrer_klasse!}\\
			\hline
			\underline{\lstinline!klasseNR!}&\underline{\lstinline!lehrerNR!}\\
			\hline
			1&12\\
			1&13\\
			2&12\\
			2&13\\
		\end{tabular}
	\end{minipage}%
\end{minipage}
Die N:M Beziehung zwischen \lstinline!klasse! und \lstinline!lehrer! kann so nicht eingespart werden. Ein Lehrer unterrichtet im Normalfall mehrere Klassen, also müsste man in der Tabelle Klasse mehrere Einträge hinzufügen und umgekehrt wird eine Klasse von mehreren verschiedenen Lehrern unterrichtet.
\begin{Exercise}[title={Überlege dir welche Beziehungstabellen man in Aufgabe \ref{ERMErstellen1} wegoptimieren kann und gib an, welches Attribut man an welchen Entitätstyp als Fremdschlüssel hinzufügen muss.}, label=Optimierung]
\end{Exercise}
%%%%%%%%%%%%%%%%%%%%%%%%%%%%%%%%%%%%%%%%%%
\begin{Answer}[ref=Optimierung]
	\begin{itemize}
		\item Fahrradverleih:

		Die Beziehungstabelle zur Beziehung \lstinline!mietet! kann wegoptimiert werden, indem man das Attribut \lstinline!kundenNR! als Fremdschlüssel zur Tabelle \lstinline!fahrrad! hinzufügt.
		\item Autohändler:

		Die Beziehungstabelle zur Beziehung \lstinline!gekauft! kann wegoptimiert werden, indem man das Attribut \lstinline!kundenNR! als Fremdschlüssel zur Tabelle \lstinline!auto! hinzufügt.

		Die Beziehungstabelle zur Beziehung \lstinline!verfuegt! kann wegoptimiert werden, indem man das Attribut \lstinline!fahrzeugID! als Fremdschlüssel zur Tabelle \lstinline!reifen! hinzufügt.
		\item DVD-Verleih:

		Die Beziehungstabelle zur Beziehung \lstinline!arbeitet_in! kann wegoptimiert werden, indem man das Attribut \lstinline!filiale.id! als Fremdschlüssel zur Tabelle \lstinline!mitarbeiter! hinzufügt.
	\end{itemize}
\end{Answer}
	\newpage
	% !TeX root = ../Skript_DB.tex
\cohead{\Large\textbf{Datentypen}}
\subsection{Datentypen}
Eine kurze Wiederholung: Eine relationale Datenbank besteht aus verschiedenen \textcolor{red}{Entitätstypen} und \textcolor{red}{Beziehungen} zwischen diesen, die jeweils als Tabellen abgebildet werden. Die \textcolor{red}{Entitätstypen} und \textcolor{red}{Beziehungen} sind dabei sozusagen die Überschriften der Tabellen, die einzelnen Spalten bezeichnet man als \textcolor{blue}{Attribute} und die Zeilen als Entitäten:

\begin{table}[h]
	\centering
	\begin{tabular}{llll}
		\multicolumn{4}{c}{\textcolor{red}{\textbf{schüler}}}\\
		\textcolor{blue}{\underline{schNr}} 	& \textcolor{blue}{vorname} 	& \textcolor{blue}{nachname}	& \textcolor{blue}{alter}  \\
		\midrule
		23&Heinz&Huber&15\\
		24&Max&Power&NULL\\
	\end{tabular}
\end{table}
\textcolor{red}{schüler} ist hier der \textcolor{red}{Entitätstyp} mit den \textcolor{blue}{Attributen schNr, vorname, nachname} und \textcolor{blue}{alter}. Der Schüler mit der \textcolor{blue}{schNr} 23, Heinz, Huber, 15 ist eine Entität.
Datenbanken benötigten meist bereits beim Erstellen eines \textcolor{red}{Entitätstyps}/Tabelle eine Angabe zum Datentyp der jeweiligen Attribute. Wir beschränken uns hier auf wenige \glqq große\grqq{ }Datentypen. Je nach Datenbankmanagementsystem lassen sich die Datentypen nochmals in mehrere kleinere Untertypen aufspalten.

SQLite ist ein beliebtes DBMS, da es klein und relativ simpel ist. Wir werden selbst mit SQLite arbeiten, weil außerdem eine portable Version verfügbar ist, d.h. es ist keine Installation notwendig. SQLite beschränkt sich darüber hinaus bereits selbst auf wenige Datentypen:
\begin{tcolorbox}[title=Datentypen]
	\begin{itemize}
		\item INTEGER: ganze Zahlen, also 0, 1, -1, 2, -2, 3, -3,\dots
		\item REAL (oft auch float genannt): Fließkommazahlen, z.B. 1,25 oder -13,9
		\item TEXT (oft auch char oder string genannt): Text, z.B. Berufskolleg
		\item BLOB: Als sog. Binary Large Objects werden z.B. Bilder oder Videos auf Datenbanken gespeichert. Zum Betrachten oder weiteren Verarbeiten müssen andere Programme als das DBMS verwendet werden.
		\item DATE/DATETIME (Anmerkung: SQLite hat hierfür keinen eigenen Datentyp, die meisten DBMS jedoch schon. SQLite speichert ein Datum als Text oder Zahl ab): Datum bzw. Zeitstempel.
	\end{itemize}
\end{tcolorbox}

\begin{Exercise}[title={Beantworte folgende Fragen; Du kannst das Internet zu Rate ziehen.}, label=Datentypen]
	\begin{enumerate}
		\item Informiere dich über den NULL-Wert, der oben in der Datenbank vorkommt. Für was steht dieser Wert? Was ist der Unterschied zwischen Null bzw. 0 und NULL?
		\item Was ist ein Byte?
		\item Wie werden INTEGER auf der Datenbank gespeichert?
	\end{enumerate}
\end{Exercise}
%%%%%%%%%%%%%%%%%%%%%%%%%%%%%%%%%%%%%%%%%
\begin{Answer}[ref=Datentypen]
	\begin{enumerate}
		\item Informiere dich über den NULL-Wert, der oben in der Datenbank vorkommt. Für was steht dieser Wert? Was ist der Unterschied zwischen Null bzw. 0 und NULL?

		Der Wert NULL bedeutet, dass kein Wert vorhanden ist. Ein ähnliches Konzept kennen wir aus der Mathematik. Die Gleichung \(x^2=0\) hat die Lösung \(x=0\), während die Gleichung \(x^2=-1\) keine Lösung hat, was wir durch das Blitzsymbol \Lightning\normalsize anzeigen. Im obigen Beispiel steht ein Wert von 0 für das Alter für eine Person, die ihren ersten Geburtstag noch nicht hatte. Ein Wert von NULL bedeutet, dass das Alter unbekannt ist.
		\item Was ist ein Byte?

		Ein Byte ist eine Informationseinheit, die normalerweise aus 8 Bit besteht. Ein Bit kann die beiden Zustände 1 oder 0 annehmen. Ein Byte kann also \(2^8=256\) verschiedene Zustände annehmen. Ältere Zeichensätze haben jeweils ein Zeichen in ein Byte gespeichert. So konnten also 256 verschiedene Zeichen (z.B. a, b, c, A, B, C, §, +, usw.) unterschieden werden.
		\item Wie werden INTEGER auf der Datenbank gespeichert?

		Im Alltag verwenden wir das Dezimalsystem, d.h. jede Zahl wird in Form von Potenzen von 10 dargestellt:

		\(123=1\cdot10^2+2\cdot 10^1+3\cdot 10^0=100+20+3\)

		INTEGER werden einfach vom Dezimalsystem auf das Binärsystem übertragen:

		\(123=1111011_{BIN}=1\cdot2^6+1\cdot2^5+1\cdot2^4+0\cdot2^3+1\cdot2^2+1\cdot2^1+1\cdot2^0\)

		\(=64+32+16+8+2+1=123\).

		Das erste Bit kann als Vorzeichen verwendet werden. Dann kann man in einem Byte Zahlen von \(-128\) bis \(127\) speichern. Je mehr Byte man für eine Zahl verwendet, desto mehr Speicherplatz benötigt man. Jedoch lassen sich dann auch größere Zahlen speichern.
	\end{enumerate}
\end{Answer}
	\newpage
	%%%%%%%%%%%%%%%%%%%%%%%%%%%%%%%%%%%%%%%%%%%%%%%%%%%
	\rohead{DBMS und SQL}
	% !TeX root = ../Skript_DB.tex
\cohead{\Large\textbf{DBMS und SQL}}
\section[DBMS und SQL]{Datenbankmanagementsystem und Structured Query Language}
\subsection{Begriffe}
In einer Datenbank werden Daten oder auch Informationen abgelegt. Für das Erstellen, Ändern oder auch für die Abfrage von Daten benötigt man ein Programm, das sogenannte Datenbankmanagementsystem. So wie es verschiedene Tabellenkalkulationsprogramme wie z.B. Excel, Calc oder Numbers gibt, stehen  verschiedene DBMS wie z.B. Oracle, Postgres oder DB2 zur Auswahl.

Obwohl es kleinere Unterschiede vor allem im Design der Oberfläche gibt, funktionieren alle Tabellenkalkulationsprogramme in großen Teilen gleich, z.B. summiert \lstinline!SUM(A1; B3; C5)! die Werte der angegebenen Zellen zusammen. Etwas ähnliches gibt es für DBMS ebenfalls. Fast alle DBMS verwenden SQL bzw. Structured Query Language für das Verwalten der Datenbank. Will man z.B. eine neue Tabelle in der Datenbank erstellen, so kann man nicht einfach auf \textit{Neu} in einer Oberfläche klicken, so wie man z.B. in Word ein neues Dokument erstellen kann, sondern muss dem DBMS eine Anweisung in Textform als SQL-Befehl geben. Zum Beispiel erstellt folgender Befehl eine Tabelle zum Speichern von Farben in der RGB-Darstellung:

\lstinline!CREATE TABLE farben(laufendeNR INT PRIMARY KEY, bezeichnung TEXT,!

\lstinline!rotanteil INT NOT NULL, blauanteil INT NOT NULL, gruenanteil NOT NULL);!

Ein sehr entfernt ähnliches Konzept haben wir bei HTML mit den verschiedenen Tags kennengelernt.

SQL selbst ist nicht case sensitive, d.h. die Groß- und Kleinschreibung spielt keine Rolle. Jedoch hat es sich eingebürgert die SQL-Schlüsselwörter komplett in Großbuchstaben zu schreiben und die Bezeichnungen von Tabellen/Entitäten und Attributen klein zuschreiben. Jeder SQL-Befehlt endet mit einem Strichpunkt.

Wir werden als DBMS SQLite verwenden, da es nicht viel Speicherplatz braucht und es eine portable Version gibt, d.h. es ist keine Installation notwendig. SQLite kann man starten, indem man die \texttt{sqlite3.exe} startet.

\begin{figure}[h]
	\centering
	\includegraphics[]{\pics/sqliteStart.png}
	\caption*{Kommandozeile nach dem Starten von \texttt{sqlite3.exe}.}
\end{figure}
Alle SQLite-Befehle (diese sind keine SQL-Befehle) beginnen mit einem Punkt. Für Interessierte: Der Befehl \lstinline!.help! gibt eine Übersicht der möglichen Befehle. Wir werden aber nur wenige Befehle, wie z.B. \lstinline!.open bsp.db! verwenden.
	\newpage
	\cohead{\Large\textbf{Anlegen/Befüllen einer DB}}
\subsection[Anlegen/Befüllen einer DB]{SQL - Anlegen und Befüllen einer Datenbank}
Erinnerung: Wir werden das Programm SQLite als Datenbankmanagementsystem (DBMS) verwenden. Nach dem Starten von SQLite sind nur noch SQLite-spezifische Befehle, die immer mit einem Punkt beginnen oder SQL-Anweisungen erlaubt.

\subsection{Erstellen bzw. Öffnen einer Datenbank}
\begin{wrapfigure}{r}{6.91cm}
	\centering
	\includegraphics[]{\pics/createDB.png}
	\caption*{Da noch keine Tabellen angelegt sind, geben die Befehle \lstinline!.tables! und \lstinline!.schema! keine Ausgabe zurück.}
\end{wrapfigure}
Starte die \texttt{sqlite3.exe}. Mit dem Befehl \lstinline!.open #Name#.db! (z.B. \lstinline!.open meineDatenbank.db!) lässt sich eine bestehende DB öffnen bzw. neu erstellen. Im Verzeichnis, in dem auch \texttt{sqlite3.exe} liegt, sollte nun (falls zuvor noch nicht vorhanden) eine neue Datei \texttt{\#Name\#.db} erstellt worden sein.\\
Mit dem Befehl \lstinline!.tables! kann man sich alle in der DB vorhandenen Tabellen/Entitätstypen anzeigen lassen.\\
Mit dem Befehl \lstinline!.schema! kann man sich die Tabellen mit einer Liste der Attribute anzeigen lassen.
\begin{tcolorbox}[title=Datenbank öffnen]
	Starte \texttt{sqlite3.exe} und gib dann \lstinline!.open datenbankname.db! ein.
\end{tcolorbox}

\subsection{Erstellen bzw. löschen von Tabellen/Entitätstypen}
Neue Tabellen/Entitätstypen lassen sich mit dem SQL-Befehl \lstinline!CREATE TABLE! erzeugen:\\
\begin{tcolorbox}[title=Tabellen erstellen]
	\lstinline[breaklines=true]!CREATE TABLE name_tabelle (name_attribut1 datentyp1 einschränkung1, name_attribut2 datentyp2 einschränkung2, ...);!
\end{tcolorbox}
\textcolor{red}{ACHTUNG: Bei SQL-Befehlen den Strichpunkt am Ende der Zeile nicht vergessen!} Man kann SQL-Befehle auch auf mehrere Zeilen aufteilen.\\
Als Datentypen werden wir \lstinline!INT! für ganze Zahlen, \lstinline!TEXT! für Texte, \lstinline!BLOB! für Bilder u.ä., \lstinline!REAL! für Fließkommazahlen und \lstinline!DATE! für Geburtsdaten u.ä. benutzen. Dem interessierten Leser seien die restlichen Datentypen zum Selbststudium ans Herz gelegt.\\
Als Einschränkungen werden wir uns zumindest zu Beginn auf \lstinline!PRIMARY KEY! und \lstinline!NOT NULL! beschränken. \lstinline!PRIMARY KEY! markiert das Attribut als Primärschlüssel und sollte als erstes Attribut angelegt werden. \lstinline!NOT NULL! zeigt an, dass dieses Attribut beim Füllen der Tabelle mit Daten nicht leer bleiben darf. Die Angabe von Einschränkungen ist optional.\\
Bestehende Tabellen lassen sich mit dem Befehl \lstinline!DROP TABLE name_tabelle! wieder löschen:
\begin{tcolorbox}[title=Tabellen löschen]
	\lstinline[breaklines=true]!DROP TABLE name_tabelle !
\end{tcolorbox}
\begin{figure}[h]
	\centering
	\includegraphics[]{\pics/createTable_schueler.png}
	\caption*{Die Tabelle \lstinline!schueler! wurde angelegt. Die Befehle \lstinline!.tables! und \lstinline!.schema! haben nun einen Rückgabewert.}
\end{figure}
\begin{figure}[h]
	\centering
	\includegraphics[]{\pics/dropTabel_schueler.png}
	\caption*{Die Tabelle \lstinline!schueler! wurde gelöscht.}
\end{figure}

\subsection{Befüllen einer Tabelle mit Daten}
Um einer Tabelle eine neue Entität/Zeile hinzuzufügen, verwendet man den \lstinline!INSERT INTO! Befehl:\\
\begin{tcolorbox}[title=Befüllen einer Tabelle]
	\lstinline!INSERT INTO name_tabelle VALUES(wert1, wert2, wert3, ...);!
\end{tcolorbox}
\begin{itemize}
	\item Alle Attribute angegeben: Sollen für alle Attribute Werte angegeben werden, so kann man einfach die Werte als kommaseparierte Liste angeben:\\
	\lstinline!INSERT INTO schueler VALUES(1,'Heinz','Huber','01.01.2000','BKFH');!\\
	Beachte, dass Texte mit Anführungszeichen (neben dem \#-Zeichen) angegeben werden müssen.
	\item Manche Attribute ohne Wert: Hat man für manche Attribute keinen Wert zur Hand, so kann man die Entität trotzdem anlegen, indem man hinter den Namen der Entität die Liste der Attribute angibt, für die man Werte zur Hand hat:\\
	\lstinline!INSERT INTO schueler(schuelerNR, vorname, nachname, klasse) VALUES (20,!\\
	\lstinline!'Anton', 'Atonovich', 'BK11');!
\end{itemize}

\begin{Exercise}[title={Beantworte folgende Fragen mit Hilfe deiner Datenbank.}, label=Befuellen]
	\begin{enumerate}
		\item Erzeuge eine Tabelle \lstinline!schueler! mit den Attributen \lstinline!schuelerNR! als Primärschlüssel, der nicht NULL sein darf, \lstinline!name!, \lstinline!plz! und \lstinline!klasse!.
		\item Füge 2 verschiedene Schüler hinzu, die aus den Klassen BK13 und BK21 stammen.
		\item Was passiert, wenn man einen weiteren Schüler mit einer bereits vergebenen \lstinline!schuelerNR! hinzufügen will?
		\item Was passiert, wenn man einen weiteren Schüler einfügen will ohne eine \lstinline!schuelerNR! anzugeben?
		\item Wie sehen die Ausgaben von \lstinline!.tables! und \lstinline!.schema! aus?
	\end{enumerate}
\end{Exercise}
%%%%%%%%%%%%%%%%%%%%%%%%%%%%%%%%%%%%%%%%%
\begin{Answer}[ref=Befuellen]
	\begin{enumerate}
		\item Erzeuge eine Tabelle \lstinline!schueler! mit den Attributen \lstinline!schuelerNR! als Primärschlüssel, der nicht NULL sein darf, \lstinline!name!, \lstinline!plz! und \lstinline!klasse!.\\
		\lstinline[breaklines=true]!CREATE TABLE schueler(schuelerNR INT PRIMARY KEY NOT NULL, name TEXT, plz INT, klasse TEXT);!
		\item Füge 2 verschiedene Schüler hinzu, die aus den Klassen BK13 und BK21 stammen., z.B.:\\
		\lstinline!INSERT INTO schueler VALUES (1, 'Heinz Huber', 70180, 'BK13');!\\
		\lstinline!INSERT INTO schueler VALUES (2, 'Dasan Ilhan', 70567, 'BK21');!\\
		\item Was passiert, wenn man einen weiteren Schüler mit einer bereits vergebenen schuelerNR hinzufügen will?\\
		Z.B. folgender Befehl: \lstinline!INSERT INTO schueler VALUES (1, 'Alina Lutz', 70874, 'BK21');!
		Es wird ein Fehler ausgegeben: \lstinline!Runtime error: UNIQUE constraint failed: schueler.schuelerNR!\\
		Unique bedeutet einzigartig und constraint steht für Einschränkung. Der Fehler besagt also, dass beim Hinzufügen eines Schülers in der Tabelle \lstinline!schueler! der Wert des Attributs \lstinline!schuelerNR! nicht einzigartig war.
		\item Was passiert, wenn man einen weiteren Schüler mit der schuelerNR NULL einfügen will?\\
		Z.B. folgender Befehl: \lstinline!INSERT INTO schueler(name) VALUES ('Vanessa Oranbay');!. Dieser Befehl würde gerne eine Zeile in der Tabelle \lstinline!schueler! anlegen, bei der alle Einträge bis auf den \lstinline!name! den Wert NULL haben.
		Es wird ein Fehler ausgegeben: \lstinline!Runtime error: NOT NULL constraint failed: schueler.schuelerNR!\\
		Der Fehler besagt also, dass beim Hinzufügen eines Schülers in der Tabelle \lstinline!schueler! der Wert des Attributs \lstinline!schuelerNR! NULL war, was nicht erlaubt ist.
		\item Wie sehen die Ausgaben von .tables und .schema aus?\\
		\lstinline!sqlite> .table!\\
		\lstinline!schueler!\\
		\lstinline!sqlite> .schema!\\
		\lstinline!CREATE TABLE schueler(schuelerNR INT PRIMARY KEY NOT NULL,!\\
		\lstinline!name TEXT, plz INT, klasse TEXT);!
	\end{enumerate}
\end{Answer}
	\newpage
	\cohead{\Large\textbf{SELECT-Statement}}
\subsection[SELECT-Statement]{SQL - Das SELECT-Statement}\label{select}
Speichere die Datei \texttt{vieleSchueler.db} in deinem Verzeichnis und öffne die DB mit \texttt{sqlite3.exe} (\lstinline!.open vieleSchueler.db!). Mit dem Befehl \lstinline!.tables! oder \lstinline!.schema! kannst du dir die Tabellen (mit Attributen) anzeigen lassen.\\
Wie erwartet gibt es eine Tabelle \lstinline!schueler!. Um den Inhalt anzuzeigen, kannst du das SELECT-Statement von SQL verwenden:
\begin{tcolorbox}[title=SELECT-Statement]
	\lstinline!SELECT * FROM schueler;!
\end{tcolorbox}
Mit den Zusätzen \lstinline!ORDER BY nachname ASC! bzw. \lstinline!ORDER BY nachname DESC!  kannst du die Ausgabe nach einem Attribut aufsteigend (ascending) bzw. absteigend (descending) sortieren lassen. Ohne die Angabe von \lstinline!ASC! bzw. \lstinline!DESC! erfolgt die Ausgabe standardmäßig aufsteigend.

\begin{Exercise}[title={Beantworte folgende Fragen mit Hilfe deiner Datenbank und dem Internet.}, label=Select]
	\begin{enumerate}
		\item Welche Ausgabe erzeugt das Statement \lstinline!SELECT * FROM schueler;!?
		\item Wofür steht der Stern (\lstinline!*!) in obigem Statement?
		\item Welche Ausgabe erzeugt \lstinline!SELECT vorname, nachname FROM schueler;!?
		\item Finde ein Statement, um dir \lstinline!nachname!, \lstinline!plz! und \lstinline!klasse! anzeigen zu lassen.
	\end{enumerate}
\end{Exercise}
%%%%%%%%%%%%%%%%%%%%%%%%%%%%%%%%%%%%%%%%%
\begin{Answer}[ref=Select]
	\begin{enumerate}
		\item Welche Ausgabe erzeugt das Statement \lstinline!SELECT * FROM schueler;!?\\
		Das Statement gibt alle in der Tabelle vorhandenen Schüler aus:\\
		\begin{lstlisting}
			1|Anica|Nosudohein|6268|06.11.1998|BKFH
			2|Marlies|Gavofu|25361|06.01.2002|BK2
			3|Franz|Rotagateson|71296|13.01.1998|BK1
			4|Elisabeth|Kotibodoweiner|14798|20.11.2003|BK1
			5|Henni|Kitavare|22926|21.07.1999|BK2
			6|Mariana|Hewalode|23879|19.05.2004|BK2
			7|Henry|Zütuschatthein|94405|31.12.2004|BK1
			8|Fatma|Varobason|19370|08.01.2005|BK1
			9|Gundel|Culufledemeiner|97896|12.04.1996|BKFH
			10|Reinhold|Tulimattson|25821|08.08.1997|BK1
			11|Silvia|Cüwiwattemüller|88339|09.11.2001|BK2
			...\end{lstlisting}
		Anmerkung: Es wurden aus Platzgründen nicht alle Schüler hier aufgelistet.
		\item Wofür steht der Stern (\lstinline!*!) in obigem Statement?\\
		Der Stern ist eine sogenannte Wildcard. Das SELECT-Statement muss wissen, welche Attribute angezeigt werden sollen. Der Stern bedeutet, dass die Werte aller an der Tabelle vorhandenen Attribute ausgegeben werden.
		\item Welche Ausgabe erzeugt \lstinline!SELECT vorname, nachname FROM schueler;!?
		\begin{lstlisting}
			Anica|Nosudohein
			Marlies|Gavofu
			Franz|Rotagateson
			Elisabeth|Kotibodoweiner
			Henni|Kitavare
			Mariana|Hewalode
			Henry|Zütuschatthein
			Fatma|Varobason
			Gundel|Culufledemeiner
			Reinhold|Tulimattson
			Silvia|Cüwiwattemüller
			...\end{lstlisting}
		Da nun nicht mehr der Stern verwendet wurde, um die Werte aller Attribute anzuzeigen, werden nur die Werte von \lstinline!vorname! und \lstinline!nachname! angezeigt.
		\item Finde ein Statement, um dir \lstinline!nachname!, \lstinline!plz! und \lstinline!klasse! anzeigen zu lassen.\\
		\lstinline!SELECT nachname, plz, klasse FROM schueler;!
	\end{enumerate}
\end{Answer}
	\newpage
	% !TeX root = ../Skript_DB.tex
\cohead{\Large\textbf{WHERE-Klausel}}
\subsection[WHERE-Klausel]{SQL - Die WHERE-Klausel}\label{where}
Im letzten Abschnitt haben wir gelernt, wie wir alle Einträge einer ganzen Tabelle oder eines/mehrerer Attribute anzeigen lassen können. Oft sind die Tabellen jedoch so groß, dass es sehr mühsam wäre, alle Einträge von Hand zu durchsuchen. Daher kann man das SELECT-Statement mit Hilfe der WHERE-Klausel einschränken. Die WHERE-Klausel kann auch an viele andere Statements angehängt werden, wie z.B. beim später noch einzuführenden DELETE-Statement.

Der grundsätzliche Aufbau ist denkbar einfach:
\begin{tcolorbox}[title=WHERE-Klausel]
	\lstinline!SQL-STATEMENT WHERE bedingungen;!
\end{tcolorbox}
Der Teufel steckt wie so oft im Detail, hier im Aufbau der Bedingungen. Einige wichtige Beispiele:
\begin{itemize}
	\item \lstinline!=! gleich, \lstinline!<! kleiner, \lstinline!>! größer, \lstinline!<=! kleiner gleich, \lstinline!>=! größer gleich, \lstinline&!=& ungleich

	Beispiel: \lstinline!SELECT * FROM schueler where schuelerNR<=10;! gibt alle Schueler, deren \lstinline!schuelerNR! kleiner oder gleich 10 ist aus.
	\item \lstinline!IS NULL! bzw. \lstinline!IS NOT NULL!

	Testet, ob ein Attribut den Wert NULL hat oder eben nicht. Ein Test mit \lstinline|= NULL| funktioniert oft nicht so wie erhofft.

	Beispiele: 	\lstinline!SELECT * FROM schueler WHERE vorname IS NOT NULL;! gibt alle Schueler aus, für die beim Vornamen ein beliebiger Wert angegeben wurde. Auch Schüler, deren Vorname aus einem Text mit der Länge 0 Zeichen besteht, würden ausgegeben werden.

	\lstinline!SELECT * FROM schueler WHERE geburtsdatum IS NULL;! gibt alle Schüler aus, die beim Attribut \lstinline!geburtsdatum! keinen Wert eingetragen haben. (Es sollten fünf Schüler ohne Geburtsdatum vorhanden sein.)
	\item \lstinline!LIKE muster!

	Wird normalerweise für den Vergleich für Attributen mit dem Datentyp \lstinline!TEXT! verwendet. Für das Muster gibt man einen Text mit Platzhaltern an. Das Prozentzeichen steht für beliebig viele (Null oder mehr Zeichen), der Unterstrich für genau ein Zeichen. Beispiele:

	\lstinline!SELECT * FROM schueler WHERE vorname LIKE 'A%';!

	Gibt alle Schüler aus, deren \lstinline!vorname! mit einem A beginnt.

	\lstinline!SELECT * FROM schueler WHERE vorname LIKE 'A%i';!

	Gibt alle Schüler aus, deren \lstinline!vorname! mit einem A beginnt und einem i endet (eine Schülerin).
	\lstinline!SELECT * FROM schueler WHERE vorname LIKE 'A___';!

	Gibt alle Schüler aus, deren \lstinline!vorname! mit einem A beginnt und insgesamt genau vier Zeichen lang ist (2 Schüler).\\
	\lstinline!SELECT * FROM schueler WHERE nachname LIKE 'Sch___';!

	Gib alle Schüler aus, deren \lstinline!nachname! mit Sch beginnt und genau 6 Zeichen hat (ein Schüler).
	\item \lstinline!IN!

	Funktioniert wie mehrere Tests auf Gleichheit (\lstinline!=! oder \lstinline!IS!). Die Vergleichswerte werden als kommaseparierte Liste angegeben:

	\lstinline!SELECT * FROM schueler WHERE schuelerNR in (1,2,5);!

	Gibt die Schüler mit den \lstinline!schuelerNR! 1, 2 und 5 aus.

	\lstinline!SELECT * FROM schueler WHERE nachname IN ('Mayer', 'Maier');!

	Gibt alle Schüler aus, deren \lstinline!nachname! Mayer oder Maier lautet.
	\item \lstinline!BETWEEN!

	Dem \lstinline!IN! sehr ähnlich, diesmal kann man jedoch gleich auf einen ganzen Bereich prüfen:

	\lstinline!SELECT * FROM schueler WHERE schuelerNR BETWEEN 10 AND 20;!

	Gibt alle Schüler aus, deren \lstinline!schuelerNR! zwischen 10 und 20 liegen (10 und 20 jeweils eingeschlossen).
\end{itemize}
\textcolor{red}{ACHTUNG:} Beim Test auf NULL muss immer \lstinline!IS! statt \lstinline!=! verwendet werden, z.B. liefert \lstinline!SELECT * FROM schueler WHERE plz is NULL;! einen Schüler, während \lstinline!SELECT * FROM schueler WHERE plz = NULL;! kein Ergebnis aber auch keine Fehlermeldung liefert.

Es lassen sich mehrere Bedingungen mit einem \lstinline!AND! bzw. \lstinline!OR! verknüpfen:

\lstinline!SELECT * FROM schueler WHERE schuelerNR BETWEEN 10 AND 20!
\lstinline!OR nachname IN ('Mayer', 'Maier');!

Gibt alle Schüler aus, deren \lstinline!schuelerNR! zwischen 10 und 20 liegen oder deren \lstinline!nachname! Mayer oder Maier lautet.

\lstinline!SELECT * FROM schueler WHERE schuelerNR BETWEEN 10 AND 20!

\lstinline!AND nachname IN ('Mayer', 'Maier');!

Gibt alle Schüler aus, deren \lstinline!schuelerNR! zwischen 10 und 20 liegen und deren \lstinline!nachname! Mayer oder Maier lautet. (Nur eine Schülerin erfüllt beide Bedingungen gleichzeitig.)

\begin{Exercise}[title={Erstelle ein SELECT-Statement mit WHERE-Klausel, das}, label=Where]
	\begin{enumerate}
		\item den Schüler mit der \lstinline!schuelerNR! 31 zurück gibt.
		\item alle Schüler mit der \lstinline!schuelerNR! 10, 23, 50 und 65 findet.
		\item alle Schüler findet, deren \lstinline!nachname! auf hein endet.
		\item alle Schüler findet, deren \lstinline!nachname! nicht auf hein endet.
		\item den Schüler findet, für den keine \lstinline!klasse! angegeben worden ist.
		\item alle Schüler findet, deren \lstinline!schuelerNR! kleiner 18 ist.
		\item alle Schüler findet, deren \lstinline!vorname! aus genau 4 Buchstaben besteht.
		\item Bonus-Frage: Warum gibt das Statement \lstinline!SELECT * FROM schueler WHERE geburtsdatum BETWEEN '01.01.2000' AND '31.12.2000';! viel zu viele Schüler zurück?

		Tipp: Probiere das Statement \lstinline!SELECT * FROM schueler WHERE geburtsdatum BETWEEN '31.01.2000' AND '31.12.2000';! oder das Statement \lstinline!SELECT geburtsdatum FROM schueler ORDER BY geburtsdatum;! aus.
	\end{enumerate}
\end{Exercise}
%%%%%%%%%%%%%%%%%%%%%%%%%%%%%%%%%%%%%%%%%
\begin{Answer}[ref=Where]
	\begin{enumerate}
		\item den Schüler mit der \lstinline!schuelerNR! 31 zurück gibt.

		\lstinline!SELECT * FROM schueler WHERE schuelerNR = 31;!
		\item alle Schüler mit der \lstinline!schuelerNR! 10, 23, 50 und 65 findet.

		\lstinline!SELECT * FROM schueler WHERE schuelerNR in (10, 23, 50, 65);!
		\item alle Schüler findet, deren \lstinline!nachname! auf hein endet.

		\lstinline!SELECT * FROM schueler WHERE nachname LIKE '%hein';!
		\item alle Schüler findet, deren \lstinline!nachname! nicht auf hein endet.

		\lstinline!SELECT * FROM schueler WHERE nachname NOT LIKE '%hein';!
		\item alle Schüler findet, für die keine \lstinline!klasse! angegeben worden ist.

		\lstinline!SELECT * FROM schueler WHERE klasse IS NULL;!

		\item alle Schüler findet, deren \lstinline!schuelerNR! kleiner 18 ist.

		\lstinline!SELECT * FROM schueler WHERE schuelerNR < 18;!
		\item alle Schüler findet, deren \lstinline!vorname! aus genau 4 Buchstaben besteht.

		\lstinline!SELECT * FROM schueler WHERE vorname LIKE '____';!
		\item Bonus-Frage: Warum gibt das Statement \lstinline!SELECT * FROM schueler WHERE geburtsdatum BETWEEN '01.01.2000' AND '31.12.2000';! viel zu viele Schüler zurück?

		Tipp: Probiere das Statement \lstinline!SELECT * FROM schueler WHERE geburtsdatum BETWEEN '31.01.2000' AND '31.12.2000';! oder das Statement \lstinline!SELECT geburtsdatum FROM schueler ORDER BY geburtsdatum;! aus.

		Das zweite Statement aus dem Tipp zeigt, dass die Geburtsdaten nicht wie erwartet sortiert werden, sondern Zeichen für Zeichen. Da die Punkte bei allen Geburtsdaten an der gleichen Stelle stehen, können wir diese ignorieren. Man kann sich die Geburtsdaten dann einfach als Zahlen vorstellen, z.B. wäre 01.04.1995 die Zahl 1.041.995. Nun werden alle \lstinline!schueler! mit Geburtsdaten, deren Zahl zw. 1.012.000 und 31.122.000 liegen, zurückgegeben. Z.B. entspricht das Geburtsdatum der Schülerin mit \lstinline!schuelerNR! 100 10.07.1997 der Zahl 10.071.997. Da diese Zahl zwischen den beiden Grenzen 1.012.000 und 31.122.000 liegt, wird sie \textit{fälschlicherweise} ausgegeben.
	\end{enumerate}
\end{Answer}
	\newpage
	\cohead{\Large\textbf{DELETE-Statement}}
\section[DELETE-Statement]{SQL - Das DELETE-Statement}
Mit Hilfe des DELETE-Statements lassen sich Einträge aus einer Tabelle löschen (Das DROP-Statement war zum Löschen der kompletten Tabelle):
\begin{tcolorbox}[title=DELETE-Statement]
	\lstinline!DELETE FROM name_der_Tabelle WHERE bedingungen;!
\end{tcolorbox}
\textcolor{red}{ACHTUNG: Vergisst man die WHERE-Klausel oder wählt mit der WHERE-Klausel mehr Einträge aus als beabsichtigt, werden ALLE oder mehr Einträge als beabsichtigt gelöscht.}

\begin{Exercise}[title={Lösche folgende Einträge aus der Datenbank:}, label=Delete]
	\begin{enumerate}
		\item Den Schüler mit der \lstinline!schuelerNR! 40.
		\item Alle Schüler mit der \lstinline!schuelerNR! 10, 20, 50 und 60.
		\item Alle Schüler, deren \lstinline!vorname! auf ia endet.
	\end{enumerate}
\end{Exercise}
%%%%%%%%%%%%%%%%%%%%%%%%%%%%%%%%%%%%%%%%%
\begin{Answer}[ref=Delete]
	\begin{enumerate}
		\item Den Schüler mit der \lstinline!schuelerNR! 40.\\
		\lstinline!DELETE FROM schueler WHERE schuelerNR = 40;!
		\item Alle Schüler mit der \lstinline!schuelerNR! 10, 20, 50 und 60.\\
		\lstinline!DELETE FROM schueler WHERE schuelerNR IN (10, 20, 50, 60);!
		\item Alle Schüler, deren \lstinline!vorname! auf ia endet.\\
		\lstinline!DELETE FROM schueler WHERE vorname LIKE '%ia';!
	\end{enumerate}
\end{Answer}
	\cohead{\Large\textbf{UPDATE-Statement}}
\subsection[UPDATE-Statement]{SQL - Das UPDATE-Statement}
Mit Hilfe des UPDATE-Statements lassen sich einzelne Werte eines Eintrags in einer Tabelle nachträglich ändern:
\begin{tcolorbox}[title=UPDATE-Statement]
	\lstinline[breaklines=true]!UPDATE name_der_Tabelle SET attribut1=wert1, attribut2=wert2, ... WHERE bedingungen;!
\end{tcolorbox}
\textcolor{red}{ACHTUNG: Vergisst man die WHERE-Klausel oder wählt mit der WHERE-Klausel mehr Einträge aus als beabsichtigt, werden ALLE oder mehr Einträge als beabsichtigt geändert.}
\begin{Exercise}[title={Ändere folgende Einträge aus der Datenbank:}, label=Update]
	\begin{enumerate}
		\item Der Schüler mit der \lstinline!schuelerNR! 19 soll mit \lstinline!vornamen! Hans Gustav Adalbert heißen.
		\item Die Bezeichnung der \lstinline!klasse! soll von BK2 auf BK22 geändert werden.
		\item Bei einigen Schülern ist ein Problem beim Eintragen der \lstinline!plz! aufgetreten. Bei allen Schülern mit einer 4-stelligen \lstinline!plz! soll diese auf NULL geändert werden.
	\end{enumerate}
\end{Exercise}
%%%%%%%%%%%%%%%%%%%%%%%%%%%%%%%%%%%%%%%%%
\begin{Answer}[ref=Update]
	\begin{enumerate}
		\item Der Schüler mit der \lstinline!schuelerNR! 19 soll mit \lstinline!vornamen! Hans Gustav Adalbert heißen.\\
		\lstinline!UPDATE schueler SET vorname = 'Hans Gustav Adalbert' WHERE schuelerNR = 19;!
		\item Die Bezeichnung der \lstinline!klasse! soll von BK2 auf BK22 geändert werden.\\
		\lstinline!UPDATE schueler SET klasse = 'BK22' WHERE klasse IS 'BK2';!
		\item Bei einigen Schülern ist ein Problem beim Eintragen der \lstinline!plz! aufgetreten. Bei allen Schülern mit einer 4-stelligen \lstinline!plz! soll diese auf NULL geändert werden.\\
		\lstinline!UPDATE schueler SET plz = NULL WHERE plz < 10000;!
	\end{enumerate}
\end{Answer}
	\newpage
	\cohead{\Large\textbf{Funktionen}}
\section[Funktionen]{Funktionen im SELECT-Statement}
Öffne die Datenbank \texttt{schule.db} mit dem Datenbank-Browser oder sqlite3.exe.\\
SQL bietet einige einfache Funktionen an, die man innerhalb des SELECT-Statements verwenden kann:
\begin{tcolorbox}[title=Funktionen in SQL]
	\lstinline!SELECT FUNKTIONSNAME(name_attribut) FROM name_tabelle (WHERE bedingungen);!
\end{tcolorbox}
\begin{itemize}
	\item \lstinline!COUNT!	Anzahl der Werte zählen
	\item \lstinline!MAX!	Maximalwert bestimmen
	\item \lstinline!MIN!	Minimalwert bestimmen
	\item \lstinline!AVG!	Durchschnittswert bestimmen
	\item \lstinline!SUM!	Werte summieren
\end{itemize}

Offensichtlich kann man bis auf die COUNT-Funktion als Argument sinnvoll nur Attribute mit einer Zahl als Datentyp anwenden.\\
Verwendung als SQL-Statements:
\begin{itemize}
	\item \lstinline!SELECT COUNT(schuelerNR) FROM schueler;!\\
	Gibt die Anzahl der Einträge in der Tabelle schueler zurück.
	\item \lstinline!SELECT COUNT(schuelerNR) FROM schueler WHERE plz=70806;!\\
	Gibt die Anzahl der Schüler zurück, die die plz 70806 haben.
	\item \lstinline!SELECT COUNT(lehrerNR) FROM lehrer WHERE nachname LIKE 'S%';!\\
	Gibt die Anzahl an Lehrern zurück, deren Nachname mit einem S beginnt.
\end{itemize}
Mit dem Zusatz \lstinline!GROUP BY! kann Datensätze, die in einem Attribut übereinstimmen, zusammenfassen:
\begin{itemize}
	\item \lstinline!SELECT COUNT(lehrerNR) FROM unterrichtet;!
	Gibt die Anzahl der Einträge in der Tabelle \lstinline!unterrichtet! zurück.
	\item \lstinline!SELECT klasseNR, COUNT(lehrerNR) FROM unterrichtet GROUP BY klasseNR;!
	Gibt die Anzahl verschiedener Lehrer (eigentlich die Anzahl von Einträgen pro Klasse) für jede Klasse bzw. \lstinline!klasseNR! zurück.
\end{itemize}

\begin{Exercise}[title={Bearbeite folgende Aufgaben}, label=Funktionen]
	\begin{enumerate}
		\item Erstelle ein zur Datenbank passendes ERM. Tipp: \lstinline!.schema! könnte hilfreich sein.
		\item Wie viele verschiedene Lehrer unterrichten an der Schule?
		\item Wie viele Schüler kommen aus einer Stadt, deren PLZ mit einer 7 beginnt?
		\item Wie viele Lehrer sind den einzelnen Abteilungen jeweils zugeordnet?
	\end{enumerate}
\end{Exercise}
%%%%%%%%%%%%%%%%%%%%%%%%%%%%%%%%%%%%%%%%%
\begin{Answer}[ref=Funktionen]
	\begin{enumerate}
		\item Erstelle ein zur Datenbank passendes ERM. Tipp: \lstinline!.schema! könnte hilfreich sein.\\
		\includegraphics[width=\linewidth]{\pics/schuleDB.png}
		\item Wie viele verschiedene Lehrer unterrichten an der Schule?\\
		\lstinline!SELECT COUNT(lehrerNR) FROM lehrer;!\\
		Es sind 17 Lehrer.
		\item Wie viele Schüler kommen aus einer Stadt, deren PLZ mit einer 7 beginnt?\\
		\lstinline!SELECT COUNT(schuelerNR) FROM schueler WHERE plz between 70000 AND 79999;!\\
		Es sind 142 Schüler.
		\item Wie viele Lehrer sind den einzelnen Abteilungen jeweils zugeordnet?\\
		\lstinline!SELECT abteilungNR, COUNT(lehrerNR) FROM lehrer_abteilung GROUP BY abteilungNR;!\\
		\begin{lstlisting}
			3|6
			6|7
			8|3\end{lstlisting}
		\lstinline!SELECT abteilungNR, bezeichnung FROM abteilung;!\\
		\begin{lstlisting}
			3|WG
			6|Berufskolleg
			8|Berufsschule\end{lstlisting}
		Das bedeutet also, dass im WG 6 Lehrer, im Berufskolleg 7 Lehrer und in der Berufsschule 3 Lehrer unterrichten.
	\end{enumerate}
\end{Answer}
	\newpage
	% !TeX root = ../Skript_DB.tex
\cohead{\Large\textbf{JOIN-Statement}}
\subsection[JOIN-Statement]{SQL - Das JOIN-Statement}\label{join}
Öffne die Datenbank \texttt{schuleOptimiert.db} mit dem Datenbank-Browser oder sqlite3.exe.

Die Datenbank beinhaltet die gleichen Informationen wie \texttt{schule.db}, jedoch wurde die Datenbank dahingehend optimiert, dass alle zu-1-Beziehungen keine eigene Beziehungstabelle mehr haben. Dies wird das Erstellen der JOIN-Statements erleichtern.
Bisher haben sich unsere SQL-Abfragen immer auf eine einzelne Tabelle bezogen. Viele Fragen lassen sich damit aber nicht oder nicht zufriedenstellend beantworten, z.B. gibt die Tabelle \lstinline!lehrer! indirekt über die \lstinline!abteilungNR! an, welche Lehrer welcher Abteilung zugeordnet sind. Der Anwender möchte jedoch gerne die Zuordnung der Lehrer-Namen zu den Abteilungsbezeichnungen haben, statt der Zuordnung zur \lstinline!abteilungNR!. Dies kann man mit Hilfe des JOIN-Zusatzes erreichen:
\begin{tcolorbox}[title=JOIN-Statement]
	\lstinline!SELECT tabelle1.attribut1, tabelle2.attribut2, usw. FROM tabelle1!

	\lstinline!INNER JOIN tabelle2 ON tabelle1.attributx=tabelle2.attributy, usw.;!
\end{tcolorbox}
Nach dem \lstinline!SELECT! gibt man entweder \lstinline!*! für alle Attribute oder  eine Liste von Attributen an. Neu ist, dass man zwischen den Tabellen unterscheiden muss, z.B. steht \lstinline!lehrer.vorname! für das Attribut \lstinline!vorname! aus der Tabelle \lstinline!lehrer!, während \lstinline!schueler.vorname! für das Attribut \lstinline!vorname! aus der Tabelle \lstinline!schueler! steht.

\lstinline!FROM tabelle1 INNER JOIN tabelle2! gibt die beiden Tabellen an, die man verknüpfen will.

\lstinline!ON bedingungen! gibt an, welche Zeilen der beiden Tabellen als eine ausgegeben werden sollen.

\textcolor{red}{ACHTUNG: Ohne \lstinline!ON bedingungen! wird die Potenzmenge gebildet, d.h. jede Zeile der ersten Tabelle wird jeweils mit jeder Zeile der zweiten Tabelle zu jeweils einer Zeile zusammengefasst und ausgegeben, was bei großen Datenbanken zu langen Bearbeitungszeiten führt.}

Beispiel:

\lstinline!SELECT vorname, nachname, abteilungNR FROM lehrer;!
Gibt die Zuordnung von Lehrern zu Abteilungen an Hand der \lstinline!abteilungNR! an. Diese Darstellung ist abstrakt. Um die Namen der Lehrer und die Bezeichnungen der Abteilungen zu kennen, müssten wir in einer weiteren Tabellen nachschauen. Mit \lstinline!SELECT abteilungNR, bezeichnung FROM abteilung;! können wir der \lstinline!abteilungNR! die Bezeichnung zuordnen und händisch prüfen, welcher Lehrer in welcher Abteilung ist. Bei großen Datensätzen wird dies schnell mühsam.

Stattdessen können wir folgendes JOIN-Statement verwenden:

\lstinline!SELECT * FROM lehrer INNER JOIN abteilung!

\lstinline!ON lehrer.abteilungNR = abteilung.abteilungNR;!
Gibt jeweils in einer Zeile einen Eintrag aus \lstinline!lehrer! und \lstinline!abteilung! aus, so dass die \lstinline!abteilungNR! von beiden Einträgen übereinstimmen.

Das DBMS nimmt also die erste Zeile aus \lstinline!lehrer! in die Hand, liest die \lstinline!abteilungNR! aus und geht dann in die Tabelle \lstinline!abteilung!. Es prüft für jede Zeile aus \lstinline!abteilung!, ob die beiden Nummern gleich sind und falls ja, klebt es die Zeilen aus \lstinline!lehrer! und \lstinline!abteilung! nebeneinander, z.B. lautet eine Zeile des Ergebnisses:

\lstinline!5|Olaf|Scholz|3|3|WG|Scholz!

Der erste Teil \lstinline!5|Olaf|Scholz|3! stammt aus \lstinline!lehrer! mit den Attributen \lstinline!lehrer.lehrerNR=5!, \lstinline!lehrer.vorname=Olaf!, \lstinline!lehrer.nachname=Scholz! und \lstinline!lehrer.abteilungNR=3!. Der zweite Teil \lstinline!3|WG|Scholz! stammt aus \lstinline!abteilung! mit den Attributen \lstinline!abteilung.abteilungNR=3!, \lstinline!abteilung.bezeichnung=WG! und \lstinline!abteilung.abteilungsleiter=Scholz!. Diese beiden Teile wurden zu einer Zeile zusammengefasst, weil die \lstinline!abteilungNR! bei beiden 3 ist.

Übersichtlicher ist es, wenn man sich nur die Informationen ausgeben lässt, die relevant sind:

\lstinline!SELECT lehrer.vorname, lehrer.nachname, abteilung.bezeichnung FROM lehrer!

\lstinline!INNER JOIN abteilung ON lehrer.abteilungNR = abteilung.abteilungNR;!

liefert als erste Zeile \lstinline!Olaf|Scholz|WG!.

\begin{Exercise}[title={Bearbeite folgende Aufgaben}, label=Join]
	\begin{enumerate}
		\item Erzeuge eine Ausgabe, die dem Vor- und Nachnamen der Lehrer jeweils die passenden Abteilungsbezeichnungen zuordnet.
		\item Erzeuge eine Ausgabe, die dem Vor- und Nachnamen aller Schüler jeweils die Bezeichnung der passenden Klasse zuordnet.
		\item Erzeuge eine Ausgabe, die jeder Klassenbezeichnung die Anzahl der Schüler der Klasse zuordnet. Tipp: COUNT-Funktion verwenden.
		\item Erzeuge eine Ausgabe, die dem Vor- und Nachnamen aller Schüler jeweils den passenden Schultyp zuordnet. Tipp: Man muss zwei JOIN-Statements verwenden.
	\end{enumerate}
\end{Exercise}
%%%%%%%%%%%%%%%%%%%%%%%%%%%%%%%%%%%%%%%%%
\begin{Answer}[ref=Join]
	\begin{enumerate}
		\item Erzeuge eine Ausgabe, die dem Vor- und Nachnamen der Lehrer jeweils die passenden Abteilungsbezeichnungen zuordnet.

		\lstinline!SELECT lehrer.vorname, lehrer.nachname, abteilung.bezeichnung FROM lehrer INNER JOIN abteilung ON lehrer.abteilungNR = abteilung.abteilungNR;!
		\item Erzeuge eine Ausgabe, die dem Vor- und Nachnamen aller Schüler jeweils die Bezeichnung der passenden Klasse zuordnet.

		\lstinline!SELECT schueler.vorname, schueler.nachname, klasse.bezeichnung FROM schueler INNER JOIN klasse ON schueler.klasseNR = klasse.klasseNR;!
		\item Erzeuge eine Ausgabe, die jeder Klassenbezeichnung die Anzahl der Schüler der Klasse zuordnet. Tipp: COUNT-Funktion verwenden.

		\lstinline!SELECT klasse.bezeichnung, COUNT(klasse.bezeichnung) FROM schueler INNER JOIN klasse ON schueler.klasseNR = klasse.klasseNR GROUP BY klasse.bezeichnung;!
		\item Erzeuge eine Ausgabe, die dem Vor- und Nachnamen aller Schüler jeweils den passenden Schultyp zuordnet. Tipp: Man muss zwei JOIN-Statements verwenden.

		\lstinline!SELECT schueler.vorname, schueler.nachname, abteilung.bezeichnung FROM schueler INNER JOIN klasse ON schueler.klasseNR = klasse.klasseNR INNER JOIN abteilung ON klasse.abteilungNR = abteilung.abteilungNR;!
	\end{enumerate}
\end{Answer}
	\newpage

%\begin{tcolorbox}[title=]
%
%\end{tcolorbox}

%\cohead{\Large\textbf{ERM}}
%\section[Enity-Relationship-Modell]{Enity-Relationship-Modell}




\newpage
\cohead{\Large\textbf{Lösungen}}
\rohead{Lösungen}
\section{Lösungen der Aufgaben}
\shipoutAnswer
\newpage
\cohead{\Large\textbf{Anhang}}
\rohead{Anhang}
\section{Anhang: Im Skript verwendete Datenbanken}
\input{Anhang.tex}
\end{document}



%\begin{Exercise}[title={xxxx}, label=xxxx]\\
%xxxx
%\end{Exercise}
%\newpage
%\begin{Answer}[ref=xxx]\\
%xxxx
%\end{Answer}
%\begin{Exercise}[title=xxxx}, label=xxxx]\\
%xxxx
%\end{Exercise}
%\newpage
%\begin{Answer}[ref=xxx]\\
%xxxx
%\end{Answer}


%\begin{figure}[h]
%	\centering
%	\includegraphics[width=0.8\textwidth]{\quadFkt/pics/steigungswinkel.png}
%\end{figure}