\cohead{\Large\textbf{DELETE-Statement}}
\subsection[DELETE-Statement]{SQL - Das DELETE-Statement}
Mit Hilfe des DELETE-Statements lassen sich Einträge aus einer Tabelle löschen (Das DROP-Statement war zum Löschen der kompletten Tabelle):
\begin{tcolorbox}[title=DELETE-Statement]
	\lstinline!DELETE FROM name_der_Tabelle WHERE bedingungen;!
\end{tcolorbox}
\textcolor{red}{ACHTUNG: Vergisst man die WHERE-Klausel oder wählt mit der WHERE-Klausel mehr Einträge aus als beabsichtigt, werden ALLE oder mehr Einträge als beabsichtigt gelöscht.}

\begin{Exercise}[title={Lösche folgende Einträge aus der Datenbank:}, label=Delete]
	\begin{enumerate}
		\item Den Schüler mit der \lstinline!schuelerNR! 40.
		\item Alle Schüler mit der \lstinline!schuelerNR! 10, 20, 50 und 60.
		\item Alle Schüler, deren \lstinline!vorname! auf ia endet.
	\end{enumerate}
\end{Exercise}
%%%%%%%%%%%%%%%%%%%%%%%%%%%%%%%%%%%%%%%%%
\begin{Answer}[ref=Delete]
	\begin{enumerate}
		\item Den Schüler mit der \lstinline!schuelerNR! 40.\\
		\lstinline!DELETE FROM schueler WHERE schuelerNR = 40;!
		\item Alle Schüler mit der \lstinline!schuelerNR! 10, 20, 50 und 60.\\
		\lstinline!DELETE FROM schueler WHERE schuelerNR IN (10, 20, 50, 60);!
		\item Alle Schüler, deren \lstinline!vorname! auf ia endet.\\
		\lstinline!DELETE FROM schueler WHERE vorname LIKE '%ia';!
	\end{enumerate}
\end{Answer}