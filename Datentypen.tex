\cohead{\Large\textbf{Datentypen}}
\section{Datentypen}
Eine relationale Datenbank besteht aus verschiedenen \textcolor{red}{Entitätstypen} und \textcolor{red}{Beziehungen} zwischen diesen, die jeweils als Tabellen abgebildet werden. Die \textcolor{red}{Entitätstypen} und \textcolor{red}{Beziehungen} sind dabei sozusagen die Überschriften der Tabellen, die einzelnen Spalten bezeichnet man als \textcolor{blue}{Attribute} und die Zeilen als Entitäten:\\
\begin{table}[h]
	\centering
	\begin{tabular}{llll}
		\multicolumn{4}{c}{\textcolor{red}{\textbf{Schüler}}}\\
		\textcolor{blue}{schNr} 	& \textcolor{blue}{vorname} 	& \textcolor{blue}{nachname}	& \textcolor{blue}{alter}  \\
		\midrule
		23&Heinz&Huber&15\\
		24&Max&Power&NULL\\
	\end{tabular}
\end{table}
\textcolor{red}{Schüler} ist hier der \textcolor{red}{Entitätstyp} mit den \textcolor{blue}{Attributen schNr, vorname, nachname} und \textcolor{blue}{alter}. Der Schüler mit der \textcolor{blue}{schNr} 23, Heinz, Huber, 15 ist eine Entität.
Datenbanken benötigten meist bereits beim Erstellen eines \textcolor{red}{Entitätstyps}/Tabelle eine Angabe zum Datentyp der jeweiligen Attribute. Wir beschränken uns hier auf wenige "große" Datentypen. Je nach Datenbankmanagementsystem lassen sich die Datentypen nochmals in mehrere kleinere Untertypen aufspalten.
SQLite ist ein beliebtes DBMS, da es klein und relativ simpel ist. Wir werden selbst mit SQLite arbeiten, weil außerdem eine portable Version gibt. SQLite beschränkt sich darüber hinaus bereits selbst auf wenige Datentypen:
\begin{tcolorbox}[title=Datentypen]
	\begin{itemize}
		\item INTEGER (ganze Zahl)
		\item REAL (oft auch float genannt, Flließkommazahl)
		\item TEXT (oft auch char oder string genannt)
		\item BLOB
		\item DATE/DATETIME (Anmerkung: SQLite hat hierfür keinen eigenen Datentyp, die meisten DBMS jedoch schon. SQLite speichert ein Datum als Text oder Zahl ab)
	\end{itemize}
\end{tcolorbox}


\begin{Exercise}[title={Beantworte folgende Fragen; Du kannst das Internet zu Rate ziehen.}, label=Datentypen]
	\begin{enumerate}
		\item Informiere dich über den NULL-Wert, der oben in der Datenbank vorkommt. Für was steht dieser Wert? Was ist der Unterschied zwischen Null bzw. 0 und NULL?
		\item Was ist ein Byte?
		\item Wie werden INTEGER auf der Datenbank gespeichert?
	\end{enumerate}
\end{Exercise}
%%%%%%%%%%%%%%%%%%%%%%%%%%%%%%%%%%%%%%%%%
\begin{Answer}[ref=Datentypen]
	\begin{enumerate}
		\item Informiere dich über den NULL-Wert, der oben in der Datenbank vorkommt. Für was steht dieser Wert? Was ist der Unterschied zwischen Null bzw. 0 und NULL?\\
		Der Wert NULL bedeutet, dass kein Wert vorhanden ist. Ein ähnliches Konzept kennen wir aus der Mathematik. Die Gleichung \(x^2=0\) hat die Lösung \(x=0\), während die Gleichung \(x^2=-1\) keine Lösung hat, was wir durch das Blitzsymbol \Lightning\normalsize anzeigen. Im obigen Beispiel steht ein Wert von 0 für das Alter für eine Person, die ihren ersten Geburtstag noch nicht hatte. Ein Wert von NULL bedeutet, dass das Alter unbekannt ist.
		\item Was ist ein Byte?\\
		Ein Byte ist eine Informationseinheit, die normalerweise aus 8 Bit besteht. Ein Bit kann die beiden Zustände 1 oder 0 annehmen. Ein Byte kann also \(2^8=256\) verschiedene Zustände annehmen. Ältere Zeichensätze haben jeweils ein Zeichen in ein Byte gespeichert. So konnten also 256 verschiedene Zeichen (z.B. a, b, c, A, B, C, §, +, usw.) unterschieden werden.
		\item Wie werden INTEGER auf der Datenbank gespeichert?\\
		Im Alltag verwenden wir das Dezimalsystem, d.h. jede Zahl wird in Form von Potenzen von 10 dargestellt:\\
		\(123=1\cdot10^2+2\cdot 10^1+3\cdot 10^0=100+20+3\)\\
		INTEGER werden einfach vom Dezimalsystem auf das Binärsystem übertragen:\\
		\(123=1111011_{BIN}=1\cdot2^6+1\cdot2^5+1\cdot2^4+0\cdot2^3+1\cdot2^2+1\cdot2^1+1\cdot2^0\)\\
		\(=64+32+16+8+2+1=123\).\\
		Das erste Bit kann als Vorzeichen verwendet werden. Dann kann man in einem Byte Zahlen von \(-128\) bis \(127\) speichern. Je mehr Byte man für eine Zahl verwendet, desto mehr Speicherplatz benötigt man. Jedoch lassen sich dann auch größere Zahlen speichern.
	\end{enumerate}
\end{Answer}