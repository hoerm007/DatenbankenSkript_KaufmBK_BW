\documentclass[a4paper,12pt, headsepline, ngerman]{scrartcl}

\usepackage{scrlayer-scrpage}
\usepackage[nodisplayskipstretch]{setspace} %vspace before/after math mode
%\setstretch{5}
\raggedbottom
\raggedright
\onehalfspacing
%\renewcommand*{\familydefault}{\sfdefault}



\pagestyle{headings} %KOMA-Script mit Kopf-Fuß-Zeilen
\usepackage{hyperref}			%Hyperlinks setzen
\usepackage[]{exercise}			%Nach hyperref einbinden!
\usepackage{babel}				%Silbentrennung mit ngerman

\usepackage{booktabs} 			% For prettier tables

\usepackage{mathtools}  		%Mathe-Paket
\usepackage{amssymb}			%Symbole
\usepackage{bbm}				%\mathbbm{N} für natürliche Zahlen o.ä.
%\usepackage{graphicx}			%Optionen für \includegraphics{imagefile}
%Solte in mathtools beinhaltet sein
\usepackage{color}				%\textcolor{blue}{text...}
\usepackage[dvipsnames]{xcolor}
%Häufig verwendetet Farben
%ForestGreen 		RGB(  0,155, 85)
%YellowOrange		RGB(250,162, 26)

\usepackage[T1]{fontenc}		%Umlaute
\usepackage[utf8]{inputenc}
\usepackage{wrapfig}
\usepackage{caption}


\usepackage{ulem}
%\uline{important} underlined text
%\uuline{urgent} double-underlined text
%\uwave{boat} wavy underline
%\sout{wrong} line struck through word
%\xout{removed} marked over like //////////
%\dashuline{dashing} dashed underline
%\dotuline{dotty} dotted underline
\usepackage{cancel}					%Durchstreichen von Dingen in Formeln
\usepackage{enumitem}				%Aufzählungen [label=\alph*)]
\setlist[enumerate]{nosep, topsep=0pt}	%Kleinere Abstände bei Aufzählungen
\setlist[itemize]{noitemsep, topsep=0pt}

\usepackage{framed}                 %Rahmen machen \begin{framed} ... \end{framed}
\usepackage{adjustbox}
\usepackage{tcolorbox} 				%Für Boxen um Text
\usepackage{array}
%Plots
\usepackage{tikz}
\usepackage{pgf}
\usepackage{pgfmath}

\usepackage{bm}						%\bm{xxx} for bold in math mode

\usepackage{geometry}
\geometry{a4paper, portrait, left=1.5cm, right=2cm, top=1cm, bottom=2cm, headsep=0.2cm, includehead, head=27.30193pt}
\usepackage{listings} %\lstinline[language=C]!while{$a || $b}!
\definecolor{codegreen}{rgb}{0,0.6,0}
\definecolor{codegray}{rgb}{0.5,0.5,0.5}
\definecolor{codepurple}{rgb}{0.58,0,0.82}
\definecolor{backcolour}{rgb}{0.95,0.95,0.92}
\definecolor{basiccolour}{rgb}{0.9,0,0.6}

\lstdefinestyle{mystyle}{
	language=SQL,
	backgroundcolor=\color{backcolour},
	commentstyle=\color{codegreen},
	keywordstyle=\color{magenta},
	numberstyle=\tiny\color{codegray},
	stringstyle=\color{codepurple},
	basicstyle=\color{basiccolour}\ttfamily,
	breakatwhitespace=false,
	breaklines=false,
	captionpos=b,
	keepspaces=false,
	extendedchars=true,
	numbers=left,
	numbersep=5pt,
	showspaces=false,
	showstringspaces=false,
	showtabs=false,
	tabsize=2,
	columns=fullflexible %erzeugt keine komischen Leerzeichen mehr, die man erst beim Kopieren sieht
}
\lstset{literate=%
	{Ö}{{\"O}}1
	{Ä}{{\"A}}1
	{Ü}{{\"U}}1
	{ß}{{\ss}}1
	{ü}{{\"u}}1
	{ä}{{\"a}}1
	{ö}{{\"o}}1
	{~}{{\textasciitilde}}1
}
\lstset{style=mystyle}

\usepackage{marvosym} %\Lightning
\usepackage{multirow}
\renewcommand{\mvchr}[1]{\Large{\mbox{\mvs\symbol{#1}}}} %\Lightning in math mode

\usetikzlibrary{intersections}

\setkomafont{headsepline}{\color{black}}

%Exercise-Paket Umbenennungen
\renewcommand{\listexercisename}{Liste der Aufgaben}%
\renewcommand{\ExerciseName}{Aufgabe}%
\renewcommand{\AnswerName}{L{\"o}sung zu Aufgabe}%
\renewcommand{\ExerciseListName}{Aufg.}%
\renewcommand{\AnswerListName}{L{\"o}sung}%
\renewcommand{\ExePartName}{Teil}%
\renewcommand{\ArticleOf}{von\ }%
%\renewcommand{\ExerciseHeaderTitle}{\ExerciseTitle}
\renewcommand{\ExerciseHeader}{%
	\textbf{\large\ExerciseHeaderDifficulty\ExerciseName\ %
	\ExerciseHeaderNB\normalsize\ExerciseHeaderTitle\ExerciseHeaderOrigin}\medskip}
\renewcommand{\AnswerHeader}{
	\medskip\textbf{L{\"o}sung zu \ExerciseName\ \ExerciseHeaderNB}\smallskip}


%Farbe für die Lösungen, die die Schüler selbst ausfüllen sollen
\definecolor{loes}{rgb}{1,1,1}
\definecolor{tcback}{rgb}{.95,.95,.95}
\definecolor{tcframe}{rgb}{.89,.15,.21}
\definecolor{loestc}{rgb}{.95,.95,.95}
%Arbeitsblatt-Modus
\definecolor{loes}{rgb}{.36,.58,.93}
\definecolor{loestc}{rgb}{1, .4, 1}
\tcbset{
%	frame code={}
%	center title,
%	left=0pt,
%	right=0pt,
%	top=0pt,
%	bottom=0pt,
	fonttitle=\large\bfseries,
	colback=tcback,
	colframe=tcframe,
%	width=\dimexpr\textwidth\relax,
%	enlarge left by=0mm,
%	boxsep=5pt,
%	arc=0pt,outer arc=0pt,
}
\tcbuselibrary{raster}

%\sqrt[\leftroot{0}\uproot{2}n]{x}
\begin{document}
	\setlength\parindent{0pt} %keine Einrückungen beim Start eines Paragraphen

	%Header
	\lohead{Datenbanken}
	%\cohead{} %im Arbeitsblatt
	\rohead{\phantom{17.04.2024}}
	\lehead{lehead}
	\cehead{cehead}
	\rehead{rehead}
	\cofoot[]{}
	\cohead{\Large\textbf{Klausur Nr. 2 - SQL}}
	\textbf{Name:}\hspace{8cm}\textbf{/42P}

    \bigskip

	Gegeben sind Ausschnitte der folgenden Tabellen \lstinline!Schüler! und \lstinline|Lehrer|. (Die Tabellen haben noch mehr Werte):
    \begin{minipage}{\textwidth}
	\adjustbox{valign=t, padding = 0ex 0ex 4ex 0ex}{\begin{tabular}{lllll}
		\multicolumn{4}{c}{\lstinline!Schüler!}\\
		\hline
		\underline{\lstinline!SNR!}&\lstinline!Name!&\lstinline!Vorname!&\lstinline!Klasse!\\
		\hline
		10&Kurz&Max&BK21\\
		15&Betz&Lisa&BK11\\
		22&Muth&Ali&BK21\\
		78&Keck&Adan&JG2\\
	\end{tabular}}%
	\adjustbox{valign=t}{\begin{tabular}{lllll}
        \multicolumn{4}{c}{\lstinline!Lehrer!}\\
        \hline
        \underline{\lstinline!LNR!}&\lstinline!Name!&\lstinline!Vorname!&\lstinline!Klassenlehrer!\\
        \hline
        101&Jäschke&Arnim&BK21\\
        105&Müller&Matthias&BK11\\
        120&Lange&Anne&\lstinline!NULL!\\
        128&Horn&Monika&JG2\\
    \end{tabular}}%
    \end{minipage}

    \begin{Exercise}[title={Gib an Hand der obigen Ausschnitte der Tabellen die Ausgabe folgender SQL-Befehle an. (14P)}, label=KA_SQL_A1]
        \begin{enumerate}[label=\alph*)]
            \item \lstinline!SELECT Vorname FROM Schüler WHERE Klasse='BK21';!
            \item \lstinline!SELECT Vorname, Name FROM Schüler WHERE Vorname LIKE 'A%';!
            \item \lstinline!SELECT * FROM Schüler WHERE Name LIKE '%c%';!
            \item \lstinline!SELECT LNR FROM Lehrer WHERE Vorname LIKE '____';!
            \item \lstinline!SELECT COUNT(SNR) FROM Schüler;!
            \item \lstinline!SELECT COUNT(SNR) FROM Schüler WHERE Klasse='BK21';!
            \item \lstinline!SELECT SNR FROM Schüler WHERE Klasse='BK21' AND SNR<20;!
        \end{enumerate}
    \end{Exercise}
	\begin{Exercise}[title={Gib jeweils den passenden SQL-Befehl an. Denke daran, dass obige Tabellen nur Ausschnitte zeigen und die Tabellen noch deutlich mehr Einträge enthalten. (15P)}, label=KA_SQL_A2]
		\begin{enumerate}[label=\alph*)]
			\item Es sollen alle Einträge aus der Tabelle \lstinline!Schüler! ausgegeben werden.
			\item Es sollen die LNR und Namen aller Lehrer ausgegeben werden, die keine Klassenlehrer sind.
			\item Frau Horn verlässt die Schule. Ihr Eintrag soll aus der Tabelle \lstinline!Lehrer! gelöscht werden.
			\item Frau Lange soll nun die Klassenlehrerin der JG2 werden.
            \item Der Schüler Heinz Gustav Bredlow kommt neu in die BK22.
		\end{enumerate}
	\end{Exercise}
	\begin{Exercise}[title={Franz hat versucht eine Tabelle namens \lstinline!unterrichtet! zu erstellen. Leider hat er die Tabelle \lstinline!uhntericht! genannt. (7P)}, label=KA_SQL_A3]
        \begin{enumerate}[label=\alph*)]
            \item Gib den SQL-Befehl an, um die Tabelle \lstinline!uhntericht! zu löschen.
            \item Gib einen SQL-Befehl an, um selbst eine Tabelle \lstinline!unterrichtet! zu erstellen mit den Attributen \lstinline!UNR! als Primärschlüssel, \lstinline!LNR! und \lstinline!Klasse!.
        \end{enumerate}
	\end{Exercise}
    \begin{Exercise}[title={Franz hat versucht einen SQL-Befehl zu erstellen, der Namen und Vornamen des Klassenlehrers der BK11 ausgibt. Finde die 3 Fehler und verbessere sie. (6P)}, label=KA_SQL_A4]

    \lstinline!SELECT Name, Vorname WHERE Klassenlehrer=BK11!
    \end{Exercise}

	\begin{Answer}[ref=KA_SQL_A1]
			\begin{enumerate}[label=\alph*)]
				\item \lstinline!SELECT Vorname FROM Schüler WHERE Klasse='BK21';!
				
				Max
				
				Ali
				\item \lstinline!SELECT Vorname, Name FROM Schüler WHERE Vorname LIKE 'A%';!
				
				Ali Muth
				
				Adnan Keck
				\item \lstinline!SELECT * FROM Schüler WHERE Name LIKE '%c%';!
				
				78 Keck Adnan JG2
				\item \lstinline!SELECT LNR FROM Lehrer WHERE Vorname LIKE '____';!
				
				120
				\item \lstinline!SELECT COUNT(SNR) FROM Schüler;!
				
				4
				\item \lstinline!SELECT COUNT(SNR) FROM Schüler WHERE Klasse='BK21';!
				
				2
				\item \lstinline!SELECT SNR FROM Schüler WHERE Klasse='BK21' AND SNR<20;!
				
				10
			\end{enumerate}
	\end{Answer}%
	\begin{Answer}[ref=KA_SQL_A2]
		\begin{enumerate}[label=\alph*)]
			\item Es sollen alle Einträge aus der Tabelle \lstinline!Schüler! ausgegeben werden.
			
			\lstinline!SELECT * FROM Schüler;!
			\item Es sollen die LNR und Namen aller Lehrer ausgegeben werden, die keine Klassenlehrer sind.
			
			\lstinline!SELECT LNR, Name FROM Lehrer WHERE Klassenlehrer IS NULL;!
			\item Frau Horn verlässt die Schule. Ihr Eintrag soll aus der Tabelle \lstinline!Lehrer! gelöscht werden.
			
			\lstinline!DELETE FROM Lehrer WHERE LNR=128;!
			\item Frau Lange soll nun die Klassenlehrerin der JG2 werden.
			
			\lstinline!UPDATE Lehrer SET Klassenlehrer='JG2' WHERE LNR=120;!
			\item Der Schüler Heinz Gustav Bredlow kommt neu in die BK22.
			
			\lstinline!INSERT INTO Schüler VALUES (79, 'Bredlow', 'Heinz Gustav', 'BK22');!
		\end{enumerate}
	\end{Answer}%
	\begin{Answer}[ref=KA_SQL_A3]
		\begin{enumerate}[label=\alph*)]
			\item Gib den SQL-Befehl an, um die Tabelle \lstinline!uhntericht! zu löschen.
			
			\lstinline!DROP TABLE uhntericht;!
			\item Gib einen SQL-Befehl an, um selbst eine Tabelle \lstinline!unterrichtet! zu erstellen mit den Attributen \lstinline!UNR! als Primärschlüssel, \lstinline!LNR! und \lstinline!Klasse!.
			
			\lstinline!CREATE TABLE unterrichtet (UNR INT PRIMARY KEY NOT NULL, LNR INT, Klasse TEXT);!
		\end{enumerate}
	\end{Answer}%
	\begin{Answer}[ref=KA_SQL_A4]
		
		\lstinline!SELECT Name, Vorname WHERE Klassenlehrer=BK11!
		
		Es fehlen die Angabe der Tabelle, die durchsucht werden soll sowie die Hochkommata um den Text BK11 und der Strichpunkt zum Abschluss des Befehls. Korrekt wäre:
		
		\lstinline!SELECT Name, Vorname FROM Lehrer WHERE Klassenlehrer='BK11';!
	\end{Answer}%
\end{document}



%\begin{Exercise}[title={xxxx}, label=xxxx]\\
%xxxx
%\end{Exercise}

%\begin{Answer}[ref=xxx]\\
%xxxx
%\end{Answer}
%\begin{Exercise}[title=xxxx}, label=xxxx]\\
%xxxx
%\end{Exercise}
%\newpage
%\begin{Answer}[ref=xxx]\\
%xxxx
%\end{Answer}


%\begin{figure}[h]
%	\centering
%	\includegraphics[width=0.8\textwidth]{\quadFkt/pics/steigungswinkel.png}
%\end{figure}